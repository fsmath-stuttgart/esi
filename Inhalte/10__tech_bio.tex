\subsubsection{Nebenfach Technische Biologie}
Technische Biologie ist eine außergewöhnliche Verbindung klassischer,
biochemischer und molekularbiologischer Inhalte mit modernen Techniken:
Im Fokus steht oft ein direkter Anwendungsbezug.
Dabei können Mathematik und Technische Biologie
in hohem Maße wechselseitig voneinander profitieren
und sogar völlig neue Forschungsansätze hervorbringen.
Die Mathematik bietet z.B.\ hervorragende Modelle,
um biologische Systeme besser zu verstehen,
während in der belebten Welt bestimmte mathematische Prinzipien
dominieren und dadurch Indizien für besonders
erfolgsversprechende Strategien liefern. 
Leider sind die Überschneidungen der Vorlesungen
nicht unerheblich - du brauchst also einiges an Ausdauer,
um dieses Nebenfach zu studieren.

\begin{center}
 \begin{tabular}{|@{}c@{}|@{}c@{}|@{}c@{}@{}|} 

   \multicolumn{1}{c}{\makebox[2.4cm]{1}} &
   \multicolumn{1}{c}{\makebox[2.4cm]{2}}  & \multicolumn{1}{c}{\makebox[2.4cm]{3}} \\[0.2cm] 

\cline{1-2} \cline{2-3}

  \multicolumn{1}{|c|}{\bf Techn.~Bio I für NF}       & \multicolumn{1}{c|}{\bf Techn.~Bio II für NF} & \multicolumn{1}{c|}{\bf Techn.~Bio III für NF}  \\
       \it 9 LP       & \it 6 LP &    \it 3 LP  \\

\cline{1-2} \cline{2-3}

 \multicolumn{1}{c}{} &  & \multicolumn{1}{c|}{\bf Bioinfo.~\& Biostat.}  \\
\multicolumn{1}{c}{} &  & \multicolumn{1}{c|}{\it 6 LP}  \\ \cline{3-3}

 \end{tabular}
\end{center}
Alle Module schließen mit einer Prüfung ab.
Die Nebenfachnote ist der Mittelwert aller Module.
Mit Ausnahme des Moduls \glqq Biophysikalische Chemie I\grqq
~benötigt man zudem zu jedem Modul einen Schein.

%%%%%%%%%%%%%%%%%%%%%%%%%%%%%%%%%%%%%%%%%%%
%% Stand: 31. August 2010
%%%%%%%%%%%%%%%%%%%%%%%%%%%%%%%%%%%%%%%%%%%
