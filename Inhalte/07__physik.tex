\subsubsection{Nebenfach Physik}

In \textit{\glqq Experimentalphysik\grqq}\ werden im Gegensatz
zu den anderen rein theoretischen Vorlesungen
auch Versuche durchgeführt, was das Ganze etwas auflockert.
Die Vorlesungen sind in der Regel
zeitlich optimal auf die Mathematik--Vorlesungen abgestimmt.
Auch hier werden Gruppenübungen angeboten,
wobei die Übungsaufgaben eher auf die Vorlesung
\glqq Höhere Mathematik für Physiker\grqq\ abgestimmt sind.
Sie greifen meist dem Stoff der Mathematik--Vorlesungen vor, 
sind aber nach einer gewissen Eingewöhnungszeit durchaus machbar,
vor allem, wenn man schon von der Schule
eine gewisse Ahnung von Physik hat.
In diesem Zusammenhang bietet es sich an,
freiwillig die Vorlesung
\glqq Mathematische Methoden der Physik\grqq\ zu besuchen,
in der Rechenmethoden erklärt werden,
die in den Mathevorlesungen erst später auftauchen. 

Wer das Nebenfach Physik gewählt hat,
muss noch ein Praktikum machen,
bei dem man in Zweiergruppen physikalische Versuche
selbständig durchführen und schriftlich auswerten muss.

%\begin{center}
%\begin{tabular}{|@{}c@{}|@{}c@{}|@{}c@{}@{}c@{}|@{}c@{}|@{}c@{}} 
%\multicolumn{1}{c}{\makebox[2.4cm]{1}}&\multicolumn{1}{c}{\makebox[2.4cm]{2}}&
%\multicolumn{1}{c}{\makebox[2.4cm]{}}&\multicolumn{1}{c}{\makebox[2.4cm]{}}&
%\multicolumn{1}{c}{\makebox[2.4cm]{5}}&\multicolumn{1}{c}{\makebox[2.4cm]{6}}\\[0.2cm]
%\cline{1-2}\cline{5-6}
%\multicolumn{2}{|c|}{\bf ExPhys I+II}&&&\multicolumn{2}{c|}{\bf Praktikum}\\
%\multicolumn{2}{|c|}{2$\times$(4+2)~|~\it15 LP}
%&&&\multicolumn{2}{c|}{0+0+4~|~\it9 LP}\\
%\cline{1-2}\cline{5-6}
%\end{tabular}
%\end{center}

\begin{center}
	\begin{tikzpicture}
		\colnr{0}{1}{1}
		\colnr{1}{1}{2}
		\colnr{4}{1}{5}
		%\colnr{5}{1}{6}
	    \doublemodul{0}{0}{ExPhys~I+II\\15 LP}
	    \modul{4}{0}{Prakt.\\9 LP}
		
    \end{tikzpicture}
\end{center}
Am Ende des Moduls \glqq Grundlagen der Experimentalphysik I\grqq\ 
steht eine dreistündige Klausur.
Die Note aus dieser Klausur ist zugleich eure Nebenfachnote.
Außerdem braucht man den Schein aus dem \glqq Physikalischen Praktikum~I\grqq.

%%%%%%%%%%%%%%%%%%%%%%%%%%%%%%%%%%%%%%%%%%%%%%%%%%%%%%%%%
%% Stand: 31. August 2010
%%%%%%%%%%%%%%%%%%%%%%%%%%%%%%%%%%%%%%%%%%%%%%%%%%%%%%%%%
