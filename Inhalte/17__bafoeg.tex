\subsection{Ausbildungsförderung: BAföG}

Studieren kostet Geld: Miete, Essen, Trinken, Fahrkarten und Bücher --
auch wenn ihr zu Hause wohnt.
Nach Berechnungen des Studentenwerkes sind dies über 500\euro\ im Monat.
Dafür müssen i.d.R.\ eure Eltern aufkommen.
Verdienen sie aber nicht genug,
so habt Ihr Anspruch auf Berufsausbildungsförderung
nach dem Berufsausbildungsförderungsgesetz, kurz BAföG.

{\large \bf Wo und wie beantragt man BAföG?}

BAföG beantragt man beim

\begin{center}\begin{tabular}{|ll|}\hline
\multicolumn{2}{|l|}{Studierendenwerk Stuttgart}\\
\multicolumn{2}{|l|}{Amt für Ausbildungsförderung}\\
Holzgartenstrasse 11   & \\
70174 Stuttgart  & \\
\multicolumn{2}{|l|}{Tel. 0711 9574-509}\\\hline
\end{tabular}\end{center}

Antragsformulare gibt online unter\\
 \verb|https://www.studierendenwerk-stuttgart.de/antragsformulare| 

{\bf Wichtig:}
Gibt man den Antrag erst im November ab,
so verfällt das BAföG für Oktober.
Ist man knapp dran, so kann man den Antrag auch direkt
beim Amt für Ausbildungsförderung in den Briefkasten werfen.
Zur Fristwahrung reicht es, das Formblatt~1 --
das ist der eigentliche Antrag -- abzugeben,
die restlichen Unterlagen kann man nachreichen.


\newpage
{\large \bf BAföG-Beratung}

Informationen zum BAföG findet man beim Vaihinger Büro der FaVeVe
\glqq Hellblaues Nilpferd\grqq~(Tel.: 685-62004)
oder beim oben angegebenen Amt für Ausbildungsförderung.

%\begin{center}\begin{tabular}{|l|l|}\hline
%Beratungsstellen                  & Beratungszeiten            \\\hline\hline
%Amt für Ausbildungsförderung    & Mo. und Do. 9--11.30 Uhr  \\\hline
%Vaihinger Fachschaftenbüro       & Mo. 10--13 Uhr             \\
%~~\glqq Hellblaues Nilpferd \grqq & oder nach Vereinbarung     \\
%                                  & (Tel.: 685-62004)           \\\hline
%%\parbox[t]{4.5cm}{Vaihinger Fachschaftenbüro \glqq Hellblaues
%%  Nilpferd\grqq}  &\parbox[t]{6cm}{ Mo. 10--13 Uhr \par oder nach Vereinbarung (Tel.: %685-2004)} \\\hline
%\end{tabular}\end{center}

\begin{center}
\includegraphics[height=21cm]
{afs/.stud.mathe/fsmath/gemeinsame_Bilder/Comics/substitute}
\end{center}


%%%%%%%%%%%%%%%%%%%%%%%%%%%%%%%%%
%%  Stand: 31. August 2010
%%%%%%%%%%%%%%%%%%%%%%%%%%%%%%%%%
