\subsubsection{Nebenfach Luft- und Raumfahrttechnik}

Die Luft- und Raumfahrttechnik (LRT)
ist eine der Ingenieurwissenschaften,
für die die Universität Stuttgart bekannt ist.
Auch das Deutsche Zentrum für Luft- und Raumfahrt
ist am Campus vertreten und kooperiert fleißig mit der Uni.
LRT ist ein eher anspruchsvolles Nebenfach.

\vspace*{-1cm}

{\small\begin{center}\begin{tabular}{|@{}c@{}|@{}c@{}|@{}c@{}|@{}c@{}|@{}c@{}|@{}c@{}} 
\multicolumn{1}{c}{\makebox[2.4cm]{1}}&\multicolumn{1}{c}{\makebox[2.4cm]{2}}&
\multicolumn{1}{c}{\makebox[2.4cm]{3}}&\multicolumn{1}{c}{\makebox[2.4cm]{4}}&
\multicolumn{1}{c}{\makebox[2.4cm]{5}}&\multicolumn{1}{c}{\makebox[2.4cm]{6}}\\[0.2cm]
\hline
\multicolumn{2}{|c|}{\bf Physik+Elektronik}&\bf Thermodyn&
\bf Strömungsl&\multicolumn{2}{c|}{\bf Prak.~Strömungssim.}\\
\multicolumn{1}{|c}{   }&\multicolumn{1}{c|}{   }&
\multicolumn{2}{c|}{    \quad {\it oder} \quad    }&
\multicolumn{1}{c}{   }&\multicolumn{1}{c|}{    \it 3 LP}\\
\cline{6-6}
\multicolumn{2}{|c|}{\it6 LP}&\it6 LP&\it6 LP&
\multicolumn{1}{c|}{\it oder }&\\
\cline{1-4}
&\bf TM~I&\bf TM~II&\multicolumn{2}{c|}{}&\bf Luftfahrt&\\
\multicolumn{1}{|c|}{}&&\multicolumn{2}{c|}{}&\it3 LP&\\
&\it6 LP&\it6 LP&\multicolumn{2}{c|}{}&\it oder&\\
\cline{2-3}
\multicolumn{4}{c|}{}&\bf Strömung&\\
\multicolumn{4}{c|}{}&\it3 LP&\\
\cline{5-5}
\end{tabular}\end{center}}

Man wählt zwischen \glqq Grundlagen der Thermodynamik~I\grqq\ (drittes Semester)
und \glqq Strömungslehre~I\grqq\ (viertes Semester).

Außerdem hat man die Wahl zwischen
\glqq Rechnerpraktikum Strömungssimulation\grqq,
\glqq Einführung in die Luftfahrttechnik\grqq\ 
und \glqq Rechnerpraktikum Numerische Simulation
von Strömung und Wärmeleitung\grqq.

%\begin{tabular}{lcrl}
%Nebenfachnote & = &$\frac{2}{7}\,\cdot$&Physik und Elektronik für LRT\\[0.5ex]
%              & + &$\frac{2}{7}\,\cdot$&Technische Mechanik I\\[0.5ex]
%              & + &$\frac{1}{7}\,\cdot$&Technische Mechanik II\\[0.5ex]
%              & + &$\frac{2}{7}\,\cdot$&Grundl.\,der\,Thermodynamik\,I {\it oder} Strömungslehre\,I\\ 
%\end{tabular}

%%%%%%%%%%%%%%%%%%%%%%%%%%%%%%%%%%%%%%%%%%%%
%%  Stand: 31. August 2010
%%%%%%%%%%%%%%%%%%%%%%%%%%%%%%%%%%%%%%%%%%%%
