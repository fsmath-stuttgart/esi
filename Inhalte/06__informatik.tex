\subsubsection{Nebenfach Informatik}

Informatik ist die Wissenschaft von den elektronischen
Datenverarbeitungsanlagen und den Grundlagen ihrer Anwendung.
Der Hauptvorteil an Informatik als Nebenfach dürfte wohl sein,
dass man später im Beruf als Ma\-the\-ma\-ti\-ker mit EDV--Kenntnissen
und Programmiererfahrung unter Umständen mehr Chancen
hat als jemand ohne diese Kenntnisse.
Auch vom Stoff her können sich Mathematik-
und Informatikvorlesungen gut ergänzen.\\[-3ex]
\begin{center}
\begin{tabular}{|@{} c @{}|@{} c @{}|@{} c @{}|@{} c @{}|@{} c @{}@{} c @{}} 
\multicolumn{1}{c}{\makebox[2.4cm]{1}}&\multicolumn{1}{c}{\makebox[2.4cm]{2}}& 
\multicolumn{1}{c}{\makebox[2.4cm]{ }}&\multicolumn{1}{c}{\makebox[2.4cm]{4}}&
\multicolumn{1}{c}{\makebox[2.4cm]{ }}&\multicolumn{1}{c}{\makebox[2.4cm]{ }}\\[0.2cm] 
\cline{1-2}\cline{4-4}
\bf Prog.+Soft.&\bf Daten+Alg.&&\bf Formale~Spr.&&\\
4+2~|~\it9 LP&4+2~|~\it9 LP&&3+1~|~6 LP&&\\
\cline{1-2}\cline{4-4}
\end{tabular}
\end{center}

Alle Module schließen mit einer schriftlichen Prüfung ab. Es gilt:\\[0.5ex]
\begin{tabular}{lcrl}
Nebenfachnote & = &$\frac{9}{24}\,\cdot$&Programmierung und Softwaretechnik\\[0.5ex]
              & + &$\frac{9}{24}\,\cdot$&Datenstrukturen und Algorithmen\\[0.5ex]
              & + &$\frac{6}{24}\,\cdot$&Automaten und Formale Sprachen\\
&&& (Theoretische Grundlagen)\\ 
\end{tabular}

%%%%%%%%%%%%%%%%%%%%%%%%%%%%%%%%%%%%%%%%%%%%
%%  Stand: 31. August 2010
%%%%%%%%%%%%%%%%%%%%%%%%%%%%%%%%%%%%%%%%%%%%
