\subsubsection{Nebenfach Technische Kybernetik}

Kybernetik ist - als Teilgebiet der Systemwissenschaft
- die Lehre der Kommunikation und Regelung
von lebenden Organismen und Maschinen und
wird auch als die Kunst des Steuerns bezeichnet.
Die Kybernetik erforscht die grundlegenden Konzepte
zur Steuerung und Regulation von Systemen und
hat einen starken mathematischen Bezug.
Durch die interdisziplinäre Ausrichtung
bietet sich der Kybernetik ein vielfältig und
breitgefächertes Anwendungsspektrum,
das Absolventen hervorragende Berufsmöglichkeiten bietet.



%\begin{center}
% \begin{tabular}{|@{}c@{}|@{}c@{}|@{}c@{}|@{}c@{}|@{}c@{}|@{}c@{}} 
%
%   \multicolumn{1}{c}{\makebox[2.4cm]{1}} &
%   \multicolumn{1}{c}{\makebox[2.4cm]{2}}  & \multicolumn{1}{c}{\makebox[2.4cm]{3}} &
%   \multicolumn{1}{c}{\makebox[2.4cm]{4}} &
%   \multicolumn{1}{c}{\makebox[2.4cm]{5}}  & \multicolumn{1}{c}{\makebox[2.4cm]{}} \\[0.2cm] 
%
%\cline{1-5}
%
%  \multicolumn{2}{|c|}{\bf ExPhys I+II}                &\bf Projekt &\bf S.dynamik&\bf Regelungst.&  \\
%  \multicolumn{1}{|c}{3+2}& \multicolumn{1}{c|}{4+2}& \it 3 LP& \it 3 LP& \it 3 LP&  \\
%\cline{3-5}
%  \multicolumn{2}{|c|}{\it 15 LP}                   &\multicolumn{4}{c}{}  \\
% \cline{1-2}
% \end{tabular}
%\end{center}

\begin{center}
	\begin{tikzpicture}
		\colnr{0}{1}{1}
		\colnr{1}{1}{2}
		\colnr{2}{1}{3}
		\colnr{3}{1}{4}
		\colnr{4}{1}{5}
	    \doublemodul{0}{0}{ExPhys~I+II\\15 LP}
	    \modul{2}{0}{Proj.\\3 LP}
		\modul{3}{0}{SD\\3 LP}
		\modul{4}{0}{RT\\3 LP}
    \end{tikzpicture}
\end{center}
Von der \glqq Projektarbeit Technische Kybernetik\grqq~(meist Teilnahme am Roborace) benötigt man einen Schein, alle anderen Module schließen mit einer Prüfung ab. Es gilt:\\[0.5ex]
\begin{tabular}{lcrl}
Nebenfachnote & = &$\frac{15}{27}\,\cdot$ &Grundlagen der Experimentalphysik~I\\[0.5ex]
              & + &$\frac{3}{27}\,\cdot$ &Projektarbeit\\ [0.5ex]
              & + &$\frac{3}{27}\,\cdot$ &Systemdynamik (SD)\\[0.5ex]
              & + &$\frac{3}{27}\,\cdot$ &Einführung in die Regelungstechnik (RT)\\ 
\end{tabular}

%%%%%%%%%%%%%%%%%%%%%%%%%%%%%%%%%%%%%%%%%%
%%  Stand: 31. August 2010
%%%%%%%%%%%%%%%%%%%%%%%%%%%%%%%%%%%%%%%%%%
