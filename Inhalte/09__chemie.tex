\subsubsection{Nebenfach Chemie}

Chemie gibt es erst seit dem Bachelor als Nebenfach.
Da es sich aber im ersten Semester
massiv mit Mathematik-Vorlesungen überschneidet,
gibt es leider bis jetzt keine uns bekannten Erfahrungen.\\[-3ex]

\begin{center}\begin{tabular}{|@{}c@{}|@{}c@{}|@{}c@{}@{}c@{}|@{}c@{}|@{}c@{}} 
\multicolumn{1}{c}{\makebox[2.4cm]{1}}&\multicolumn{1}{c}{\makebox[2.4cm]{2}}&
\multicolumn{1}{c}{\makebox[2.4cm]{}}&\multicolumn{1}{c}{\makebox[2.4cm]{}}&
\multicolumn{1}{c}{\makebox[2.4cm]{5}}&\multicolumn{1}{c}{\makebox[2.4cm]{}}\\[0.2cm] 
\cline{1-2}\cline{5-5}
\bf Einf.~Chemie&\bf Prak.~Chemie&&&\bf Theo.~Chemie&\\
6+3~|~\it12 LP&6P~|~\it6 LP&&&3+1~|~\it6 LP&\\
\cline{1-2}\cline{5-5}
\end{tabular}\end{center}
Aus dem Modul \glqq Einführung in die Praktische Chemie\grqq
~benötigt man einen Schein,
die anderen beiden Module haben Prüfungen.
Es gilt:\\[0.5ex]
\begin{tabular}{lcrl}
Nebenfachnote & = &$\frac{12}{18}\,\cdot$&Einführung in die Chemie\\[0.5ex]
              & + &$\frac{6}{18}\,\cdot$&Theoretische Chemie\\ 
\end{tabular}


%%%%%%%%%%%%%%%%%%%%%%%%%%%%%%%%%%%%%%%%%%%%%%%%%%%%%%
%% Stand: 31. August 2010
%%%%%%%%%%%%%%%%%%%%%%%%%%%%%%%%%%%%%%%%%%%%%%%%%%%%%%
