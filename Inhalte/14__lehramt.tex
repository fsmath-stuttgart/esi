\newpage
\section{Studienplan des Bachelor of Arts für's Lehramt}\label{LA}

Den Studienplan für dein Zweitfach hier aufzuführen,
würde den Rahmen dieses Infoheftes sprengen,
darum beschränken wir uns auf die
von dir verlangten Leistungsnachweise in Mathematik.
Falls du wichtige Fragen zum Lehramtsstudium haben solltest,
gehe bitte zum Lehramtszuständigen (siehe \ref{LAzust})
oder schau bei uns in der Fachgruppe vorbei.

\subsection{Der Mathematikanteil}
Zu Beginn des Studiums wirst du folgende Grundvorlesungen hören:\\[6pt]
- {\bf Analysis~I}\\[2pt]
- {\bf Analysis~II}\\[2pt]
- {\bf Lineare Algebra und Analytische Geometrie~I} (LAAG I)\\[2pt]
- {\bf Lineare Algebra und Analytische Geometrie~II} (LAAG II)\\[6pt]
Wenn für dich Mathematik die höhere Priorität hat,
kannst du diese Vorlesungen zum Beispiel
innerhalb der ersten zwei Semester absolvieren.
Dazu müsstest du LAAG und Analysis parallel hören.
Andernfalls könntest du zunächst LAAG~I+II in den ersten zwei Semestern
und im dritten und vierten Semester Analysis~I+II hören.
Natürlich könntest du auch mit Analysis beginnen,
dies können wir dir jedoch nicht empfehlen,
da du in Analysis II die Lineare Algebra benötigst. 
Zu jedem dieser Module musst du eine schriftliche Prüfung ablegen.
Prüfungsvorleistung ist immer der jeweilige Schein.
Deine Orientierungsprüfung besteht
entweder aus der Modulprüfung zu LAAG I oder aus der Modulprüfung zu Analysis I,
diese musst du bis zum Beginn der Vorlesungszeit
des vierten Semesters bestanden haben. 
Wenn du einen Großteil der oben genannten Module absolviert hast,
kannst du dich an folgende Module machen:\\[6pt]
- {\bf Algebra und Zahlentheorie für gym. Lehramt} {\em oder}\\[2pt]
   \quad  {\bf Analysis~III}\\[2pt]
- {\bf Mathematische Programmierung für gym. Lehramt}\\[2pt]
- {\bf Stochastik und Angewandte Mathematik für gym. Lehramt}\\[2pt]
- {\bf Geometrie für gym. Lehramt}\\[2pt]
- {\bf Komplexe Analysis für gym. Lehramt}\\[2pt]
- {\bf Fachdidaktik~I}\\[6pt]

Dir ist es erlaubt zwischen Analysis~III und Algebra und Zahlentheorie zu wählen.
\begin{center}
\includegraphics[width=17cm]
{/afs/.stud.mathe/fsmath/gemeinsame_Bilder/Friederike/Lehramt1}
1. Priorität Mathematik
\end{center}
Für das Erste ist es empfohlen Analysis~I+II gehört zu haben und
für das Zweite solltest du Ahnung von LAAG~I+II haben.
Du musst nur bedenken, dass du das, was du nicht wählst, im Master hören musst.
Das Modul mathematische Programmierung setzt sich aus einem
Programmierkurs und einer größeren Programmieraufgabe zusammen.
Anschließend erkundest du die Welt der Stochastik und angewandten Mathematik. 
Um dich an Geometrie heranzuwagen, solltest du alle Grundvorlesungen hinter dir haben.
An die komplexe Analysis darfst du bereits ran, wenn du Analysis~I+II gemeistert hast.
Wie die Prüfungen genau aussehen
und ob ein Schein Prüfungsvorleistung ist,
erfährst du vom jeweiligen Dozenten zu Beginn der Veranstaltung.
Außerdem brauchst du ein Proseminar im Umfang von 3 LP.
\begin{center}
\includegraphics[width=17cm]
{/afs/.stud.mathe/fsmath/gemeinsame_Bilder/Friederike/Lehramt2}
2. Priorität Zweitfach
\end{center}
Zum Abschluss deines Bachelor of Arts ist in einem deiner beiden Fächer
eine kurze wissenschaftliche Arbeit (Bachelorarbeit) zu schreiben.
Diese hat einen Umfang von 6 LP.
%Das Modul angewandte Mathematik setzt sich zusammen
%aus einer Vorlesung mit Übung zur \glqq Wahrscheinlichkeitstheorie\grqq\ 
%(Zulassungsvoraussetzung hierfür: Analysis~I+II),
%einem Programmierkurs und eine Vorlesung zur
%\glqq Numerischen Lineare Algebra\grqq\ mit Übung
%(empfohlene Voraussetzung hierfür: LAAG~I).
%Als inhaltliche Voraussetzung für die Einführung
%in die Geometrie und Algebra
%werden LAAG~I+II empfohlen.
\subsection{Spezielle Veranstaltungen fürs Lehramt}

\subsubsection{Bildungswissenschaften}

Für Lehrämtler im Bachelor of Arts
sind einige Pädagogikveranstaltungen Pflicht.
Während deinem Studium wirst du folgende Module besuchen:\\[6pt]
- {\bf Bildungswissenschaftliche Grundlagen I:}\\
\ \ Einführung in die pädagogische Psychologie (Vorlesung, 1.~Semester)\\[2pt]
- {\bf Bildungswissenschaftliche Grundlagen II:}\\
\ \ Erziehungswissenschaftliches Arbeiten (Seminar, 1.~oder 3.~Semester)\\[2pt]
- {\bf Schulpraktische Orientierung:}\\
\ \ Analyse von Lehr- und Lernprozessen I (Seminar, 2.~Semester)\\[2pt]
- {\bf Lehren und Lernen:}\\
\ \ Didaktik (Vorlesung, 3. oder 5.~Semester)\\[2pt]
- {\bf Schulpraktische Orientierung:}\\
\ \ Orientierungspraktikum (Praktikum, 3.~oder 4.~Semester)\\[2pt]
- {\bf Lehren und Lernen:}\\
\ \ Analyse von Lehr- und Lernprozessen II (Seminar, 4.~Semester)\\[2pt]
Die Semesterangaben sind Empfehlungen,
aber in keinster Weise prüfungsrechtlich bindend.
Du solltest aber die Voraussetzungen der Module beachten:
Es ist B.W.~Grundl.~I Voraussetzung für B.W.~Grundl.~II
und Analyse von Lernprozessen~I
Voraussetzung für Analyse von Lernprozessen~II.
Es wird empfohlen, das Orientierpraktikum frühzeitig abzulegen,
denn es soll dir eine Vorstellung davon geben,
wie dein Lehrerberuf in Zukunft aussieht.

%% Im Fall des Orientierungspraktikums
%% handelt es sich um eine indiskutable Frist,
%% die unter allen Umständen eingehalten werden muss,
%% ähnlich wie die Orientierungsprüfung.

{\it Weitere ausführliche Infos/Quelle:}\\
{\small\verb|http://www.uni-stuttgart.de/studieren/angebot/lehramt/la_bama/la_bama.pdf|}

\begin{center}
\includegraphics[width=17cm]
{/afs/.stud.mathe/fsmath/gemeinsame_Bilder/Comics/certainty}
\end{center}

\subsubsection{Orientierungspraktikum}

Im Verlauf deines Studiums im Bachelor of Arts
musst du ein Orientierungspraktikum absolvieren.
Du besuchst in dieser Zeit eine Schule (nicht deine eigene)
und kannst so einen Eindruck gewinnen,
ob dir der Lehrerberuf gefallen könnte.
Das Orientierungspraktikum liefert 3 LP.\\
Zum Orientierungspraktikum muss man sich anmelden.
Nähere Informationen findest du unter:
\verb|http://www.orientierungspraktikum-bw.de/|\\
Die eigentliche Schulpraxis kommt erst im anschließenden
Master of Education.
Da wirst du ein Schulpraxismodul
von einem Umfang von 16 LP absolvieren.

{\small P.S.: Aufgrund eines Übertragungsfehlers
kursiert die Information,
das Orientierungspraktikum müsste schon
bis zu Beginn des 4.\ Semesters absolviert sein.
Dies trifft nicht zu, es wird aber trotzdem empfohlen,
das Orientierungspraktikum zeitig zu absolvieren.}

\vspace*{1cm}

%% \subsubsection{Schulpraktikum}
%% Das Schulpraktikum ist ein eigenes Modul mit 16 LP.
%% Es findet in einem ungeraden Semester (vorgesehen ist das fünfte Semester,
%% also nach der Zwischenprüfung) statt und
%% geht über 13 Wochen ab Schulbeginn im September.
%% \linebreak In dieser Zeit besuchst Du eine Schule,
%% beobachtest den Unterricht von anderen Lehrern
%% (man sagt auch \glqq hospitieren\grqq) und darfst auch
%% selbst erste Unterrichtserfahrung sammeln. \\
%% Das Schulpraxissemester kann bestanden
%% oder auch nicht bestanden werden.
%% Wenn du das Praktikum nicht bestehst,
%% kannst du es einmal wiederholen.
%% Auch zum Schulpraktikum musst du dich anmelden:\\

%% %\vspace*{0.5 cm}
%% Der Anmeldezeitraum ist vom ersten Schultag
%% nach den Osterferien bis zum 15.05., mind. jedoch 4 Wochen,
%% und erfolgt in zwei Schritten:
%% \begin{center}
%% \begin{itemize}
%%   \item
%%     Seminarplatz für das Praxissemester an einer Schule im Internet reservieren. 
%%     Eine Reservierung an mehreren Schulen ist nicht gleichzeitig möglich.
%%   \item
%%     Irgendwann meldet sich dann die Schule. Meistens muss man noch zu einem
%%     persönlichen Gespräch dorthin, das aber nur dazu dient, sich schon mal
%%     kennenzulernen.
%% \end{itemize}
%% \end{center}
%% Unter \verb|http://www.praxissemester-bw.de/| oder
%% natürlich bei uns in der Fachgruppe findest du weitere Informationen.
%% Es gibt auch die Möglichkeit, das Orientierungs-
%% und Schulpraktikum im Ausland zu absolvieren.
%% Details findest du auch dazu im Internet.
%% Wichtig ist in diesem Fall, sich frühzeitig zu informieren.




%%%%%%%%%%%%%%%%%%%%%%%%%%%%%%%%%%%%%%%%%%
%% Stand: 03. Oktober 2016
%%%%%%%%%%%%%%%%%%%%%%%%%%%%%%%%%%%%%%%%%%
