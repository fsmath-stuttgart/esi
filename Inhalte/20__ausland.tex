\newpage
\subsection{Auslandsstudium}
 
Es mag manchem vielleicht etwas verfrüht erscheinen,
aber es gibt doch einige gute Gründe sich
über ein Auslandssemester oder -jahr zu informieren,
noch bevor das Studium an der Universität Stuttgart überhaupt begonnen hat.

Zum einen muss man mit den Vorbereitungen
bis zu anderthalb Jahren vor dem Auslandssemester beginnen.
Also im zweiten Semester, falls man im fünften Semester weg will,
was vermutlich die günstigste Zeit im Rahmen des Bachelors ist.

Vor allem aber wollen wir hier Werbung machen,
diese einmalige Chance zu nutzen.
Nicht nur, dass man natürlich seine Sprachfertigkeiten verbessert,
sondern vor allem die Erfahrung einer anderen Kultur,
der Beginn neuer Freundschaften -- in aller Welt --
und die Begegnung mit einer völlig anders gearteten Universitätskultur
-- die Mathematik bleibt allerdings dieselbe --
machen dies zu einem unvergesslichen Erlebnis.
All dies bleibt denen verborgen,
die solch einen Schritt nicht wagen.

Wer mehr über ein Auslandsstudium und die Möglichkeiten
dafür an der Uni Stuttgart wissen möchte,
sollte sich beim Amt für Internationale Angelegenheiten
(Pfaffenwaldring 60) erkundigen.
Die Uni hat  zahlreiche Kontakte zu ausländischen Hochschulen.
 %%% {}\hfill{\em Bernd Ackermann} 

\vspace{1cm}

\ifthenelse{\boolean{online}}
{

}
{
\vspace{1cm}
\begin{center}
\includegraphics[width=\textwidth]
{afs/.stud.mathe/fsmath/gemeinsame_Bilder/Comics/comic5}
\end{center}
}

%%%%%%%%%%%%%%%%%%%%%%%%%%%%%%%%%%%%%%%%
%%% Stand: 30.Juni 2010
%%%%%%%%%%%%%%%%%%%%%%%%%%%%%%%%%%%%%%%%
