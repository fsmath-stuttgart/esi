\subsection{Nahverkehr}

{\large \bf Der "`VVS"' -- DAS NETZ}

(Fast) alle Busse und Bahnen des öffentlichen Personennahverkehrs
in Stuttgart und den vier umgebenden Landkreisen
gehören zum "`VVS"', dem Verkehrs- und Tarifverbund Stuttgart.

{\large \bf Mit dem Studentenausweis ab 18 Uhr durchs ganze Netz}

Vielleicht hast du dich schon gewundert,
dass du bei deiner (An- bzw.) Rückmeldung 165,60\euro\ überweisen musstest.
Von diesem Betrag gehen 45,60\euro\ an die VVS.
Dafür gilt dein gültiger Studentenausweis
(mit entsprechendem VVS-Aufdruck) täglich ab 18.00~Uhr
sowie an Wochenenden ganztags im ganzen Netz
das ganze Semester als Fahrausweis.
(Wintersemester: 1.~Okt.~- 31.~März, Sommersemster: 1.~Apr.~-30.~Sept.)
Diesen Beitrag musst du leisten,
ob du willst oder nicht bzw.~den VVS nutzt oder nicht.

{\large \bf Semesterticket (den ganzen Tag durchs ganze Netz)}

Wenn du täglich mit dem VVS zur Uni kommst,
was wahrscheinlich vor 18.00~Uhr geschieht,
gibt es eine zweite Komponente, das sogenannte "`Studiticket"'.
Hierfür musst du weitere 199,00\euro\ zahlen
und kannst dann das ganze Netz den ganzen Tag nutzen,
natürlich auch wieder das ganze Semester lang.
Dieses Ticket bekommt du bei allen SSB-Verkaufsstellen
und allen DB-Fahrkartenausgaben im VVS.
Mittlerweile gibt es auch die Möglichkeit sich dieses Ticket
online zu kaufen.
Damit hast du den Vorteil dein Ticket neu ausdrucken zu können,
falls es verloren geht.

{\large \bf Schwerbehinderte im Netz}

Studierende, die aufgrund einer Schwerbehinderung
zur kostenlosen Nutzung des Personennahverkehrs berechtigt sind,
erhalten auf Nachweis den Sockelbetrag von ca.\ 30\euro\ erstattet.

{\large \bf Verkaufsstellen und Kundenberatung}

Verkaufsstellen gibt es u.a.\ 
an der Universität direkt oben am S-Bahn-Ausgang
(der Haltestelle \glqq Universität\grqq, im runden Häuschen),
in der Klett-Passage und im Hauptbahnhof.

%\newpage
\newpage
{\it Mehr Infos bei}:\\
\\
\begin{tabular}{|l|l|l|}
\hline
VVS-Infothek & Kundenzentrum & Verkaufsstelle \\
Königsstr. 1A & Rotebühlpassage & Klettpassage (Hbf)  \\
Mo-Fr  9:00-18:30 Uhr & Mo-Fr 07:00-18:30 Uhr & Mo-Fr 07:00-19:45 Uhr \\
Sa 10:00-16:00 Uhr & & \\
\hline
\end{tabular}\\
\\
im Internet unter: \verb|http://www.vvs.de|


\vspace*{5cm}
\begin{center}
\includegraphics[width=11cm]{/afs/.stud.mathe/fsmath/gemeinsame_Bilder/Comics/comic10}
\end{center}

%%%%%%%%%%%%%%%%%%%%%%%%%%%%%%%%%%%%
%%% Stand: 23. 06. 2010
%%%%%%%%%%%%%%%%%%%%%%%%%%%%%%%%%%%%
