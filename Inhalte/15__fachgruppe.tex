\section{Die Fachgruppe Mathematik}

Was ist das: Fachgruppe\,?\\

In diesem Kapitel möchten wir dir erklären,
was die {\bf Fachgruppe} ist und was sie so alles macht.

Die Fachgruppe sieht sich als Interessenvertretung
aller Mathematik--Studenten und
betreibt ausschließlich \glqq Hochschulpolitik\grqq .
Wir sind also ganz normale Studenten,
die neben ihrem Studium noch Zeit investieren,
um für gute Studienbedingungen an unserer Universität zu sorgen.
\vspace*{1cm}
\ifthenelse{\boolean{online}}
{
}
{
\begin{center}
\includegraphics[width=\textwidth]{afs/.stud.mathe/fsmath/gemeinsame_Bilder/Comics/shoe_algebra}
\end{center}
}

Neben dieser Interessenvertretung der Studenten
wollen wir auch eine allgemeine Studienberatung sein.
{\bf Wann immer irgendwelche Probleme oder Fragen
     bezüglich des Studiums auftauchen,
     kannst du dich jederzeit an uns wenden.} 
Wir haben in der Regel einen guten Kontakt zu den Professoren
und wissen meist auch Bescheid,
welche Person in unserer Fakultät für was zuständig ist.
Vielleicht können wir nicht jede deiner Fragen beantworten,
aber zumindest können wir dich an zuständige Personen verweisen.

Hier ist eine Auf\-listung der Dinge,
die wir die ganze Zeit so machen:

\begin{itemize}

% Erstsemester
\item
Dieses {\bf Erstsemesterinformationsheft} (ESI),
das du gerade liest, wurde zum Beispiel von der
Fachgruppe zusammengestellt und geschrieben. 
Ebenso organisieren wir vor jedem Wintersemester
die {\bf Erstsemestereinführung} (ESE), um denen,
die neu an die Uni kommen und sich überhaupt noch nicht auskennen,
ein wenig mit Rat und Tat zur Seite zu stehen.
Dieses Jahr findet diese Erstsemestereinführung
am \ESETagEins. und \ESETagZwei.~Oktober statt. 
 
%Tage
\item
Die Fachgruppe beteiligt sich an vielen {\bf Informationsveranstaltungen}
(Tag der Wissenschaft, Unitag ,\dots).
Dort machen wir Werbung für das Mathematikstudium
und geben wichtige Hinweise an Schüler,
was sie alles in der Mathematik erwarten wird.
Vielleicht hast du uns ja schon einmal auf einer dieser Veranstaltungen gesehen.

%Vorlesungsumfrage
\item
Die {\bf Vorlesungsumfrage} wird von uns ausgewertet
und ist eine Möglichkeit, den Professoren
eine Rückmeldung zur ihrer Vorlesung zu geben.

%Seminarplatzvergabe
\item
Die Fachgruppe sorgt jedes Semester dafür,
dass es eine {\bf Pro- und Hauptseminarplatzvergabe}
(meistens am letzten Mittwoch in der Vorlesungszeit) gibt.
Dort werden die freien Plätze für die Seminare
des darauffolgenden Semesters vergeben.
\end{itemize}

\ifthenelse{\boolean{online}}
{
}
{
\begin{center}
\includegraphics[width=16cm]
{afs/.stud.mathe/fsmath/gemeinsame_Bilder/Comics/mathreli}
\end{center}
}

\begin{itemize}
%ROMCE
\item
Au"serdem veranstaltet die Fachgruppe für alle Mathematiker
im Sommer- und im Wintersemester das ROMCE, unser {\bf Spielewochenende},
wo wir gemeinsam ein (verlängertes) Wochenende in einer Jugendherberge verbringen.
Dieses Wochenende ist insbesondere für Erstsemester
gedacht und soll Gelegenheit bieten, sich untereinander kennenzulernen,
denn gemeinsam geht vieles besser
(siehe auch Rückseite).

%Prüfungsprotokolle
\item
Damit die Vorbereitung auf Prüfungen besser läuft,
haben wir für euch diverse {\bf Prüfungsprotokolle} von vergangenen Prüfungen.
Schriftliche Prüfungen aus den letzten Jahren
findet ihr auf unserer Internetseite (nur vom Uni-Netz aus zugänglich),
Protokolle von mündlichen Prüfungen gibt es in der Fachgruppe gegen Pfand.

\item
Die Fachgruppe verkauft  {\bf Getränke} und {\bf Sü"sigkeiten}
zum Selbstkostenpreis, damit von unseren Studenten
auch keiner verdursten muss und jeder etwas Nervennahrung für die Vorlesungen hat.

\item
Weiter könnt ihr bei uns {\bf Spiele} ausleihen.
Dort könnt ihr euch, nach Feststellung eurer Personalien,
für kurze Zeit Spiele aus unsere kleinen,
aber feinen {\bf Spielesammlung} ausleihen.
\end{itemize}

Neben den direkt sichtbaren Serviceleistungen gibt es
auch noch jede Menge von laufenden Aufgaben,
von denen viele Studenten kaum etwas merken,
die aber trotzdem wichtig sind. Dazu zählen unter anderem:

\begin{itemize}
\item
Die {\bf Fachgruppensitzung}:
Die Fachgruppe trifft sich in der Vorlesungszeit
einmal pro Woche zur Fachgruppensitzung,
zu der alle Studenten recht herzlich eingeladen sind.
In der Fachgruppensitzung wird alles besprochen,
was gerade so anliegt, z.B.\ Beteiligung an Uni-Veranstaltungen,
Organisation von Fachgruppen-Angeboten (s.o.),
Änderungen an Studienbedingungen und vieles mehr.\\
{\bf Nächster Termin:} Wird stets auf unserer Homepage angekündigt.

\item
{\bf Fachbereichs-/Fakultätsratsvertreter}:
Die Fachgruppe hat voll stimmberechtigte studentische Vertreter
im Fachbereichsrat (Mathematik) und Fakultätsrat (Mathematik/Physik),
den beiden zentralen Gremien für alle Belange des Mathematikstudiums.

\item
Die {\bf Studienkommission}:
Sie kümmert sich um die Studienordnung.
Insbesondere bei der Anpassung von Prüfungsordnungen,
aber auch bei der Verwendung von Studiengebühren
spielt sie eine wichtige Rolle.
Wir sind in diesem Gremium mit vier Leuten vertreten.

\item
Die {\bf Berufungskommissionen}:
Auch wenn es darum geht, neue Professoren auszusuchen
(um ausscheidende Professoren zu ersetzen)
haben die Studenten ein Mitspracherecht.
Wir achten darauf, dass bei den Auswahlkriterien
die Lehre nicht allzu kurz kommt
(die Professoren berücksichtigen in erster Linie die
Forschungsleistung und nicht die Qualität der Lehre).

\item
Der {\bf Fachschaftsrat} und die {\bf Stuvus}:
Die Fachgruppe entsendet Vertreter zum Fachschaftsrat,
der Interessenvertretung der Studenten der Fakultät $8$
(Mathematik und Physik).
Der Vorsitzende besitzt Stimmrecht im Studierendenparlament
der Studierendenvertretung Universität Stuttgart, kurz Stuvus. 

\newpage
Das Studierendenparlament ist das Legislativ-Organ
der Studierendenschaft der Uni Stuttgart.
Hier wird z.B.\ über Satzungen und Haushaltspläne abgestimmt,
d.\,h. unter anderem wie euer Studierendenschaftsbeitrag
von $7$\euro\ verwendet wird.
Ausserdem sind dort die uniweiten Arbeitskreise angesiedelt
z.B.\ AK Cräsh (Musikanlage), AK~Computer, AK~Umsetzbar, \dots \\
\end{itemize}

So, nun hast du einen groben Überblick darüber,
wer wir sind und was wir machen.
Wenn du Lust und Zeit hast, dann schau einfach mal vorbei,
wir freuen uns über jedes neue Gesicht. \\

Au"serdem:
Fachgruppe macht Spaß! Woran man das sieht?
- Na, warum sind wir sonst hier.

\begin{flushright}{\it Die Fachgruppe}\end{flushright}

\vspace*{3cm}
\ifthenelse{\boolean{online}}
{

}
{
\begin{center}
\includegraphics[width=8cm]
{afs/.stud.mathe/fsmath/gemeinsame_Bilder/Comics/snoopy}
\end{center}
}

%%%%%%%%%%%%%%%%%%%%%%%%%%%%%%%%%%%%%%%%
%%% Stand: 1. Juli 2010
%%%%%%%%%%%%%%%%%%%%%%%%%%%%%%%%%%%%%%%%
