\begin{center}
\includegraphics[width=\textwidth]
{afs/.stud.mathe/fsmath/gemeinsame_Bilder/Comics/titelbg1}
\end{center}

\vspace{-3cm}

\section{Studieren an einer Universität}

In diesem Heft wollen wir dir erklären,
was du alles machen musst,
um deinen Bachelor of Science
bzw.\ dein Bachelor of Arts
für's Lehramt zu erreichen.
Dabei erläutern wir dir den Ablauf
deines zukünftigen Studienalltags,
der sich unter Anderem in Vorlesungen,
Übungsgruppen, Seminare, Prüfungen am Semesterende
und so weiter gliedert...

\subsection{Das Bachelorstudium}

Das Studium für den Bachelor of Science in Mathematik
ist auf sechs Semester ausgelegt und
kann grob in zwei Abschnitte unterteilt werden.
Während der ersten drei Semester
sollst du die elementaren Grundlagen
der Mathematik erlernen,
die für fast alle darauf aufbauenden Spezialisierungen
oder gar Anwendungen unverzichtbar sind.
Auch lernst du die \glqq mathematische Art,
Wissenschaft zu betreiben
und Probleme zu lösen\grqq.

Darunter fällt die formale Beschreibung
mathematischer Sachverhalte, Beweisführung etc.,
an denen sich das Vorgehen in {\it allen}
Teilgebieten der Mathematik orientiert.
In diesem Abschnitt sind die meisten Studieninhalte fest vorgegeben.
Du kannst dir aber zum Beispiel dein Nebenfach
frei nach belieben aussuchen.
Im Gegensatz dazu kannst du dann im zweiten Teil
das meiste frei aus einem Pool an Vertiefungen auswählen.
Wo du dann schließlich deine Bachelorarbeit schreibst,
hängt von deiner Entscheidung und dem Angebot ab.

Ein vollständiges Lehramtsstudium
teilt sich auf in ein Bachelorstudium
nach dessen erfolgreichen Abschluss
du den Titel \glqq Bachelor of Arts\grqq\ erhältst,
und einen Masterstudiengang,
den du mit dem Titel \glqq Master of Education\grqq\ abschl\-ießt.
Du kannst die mathematischen Grundlagenvorlesungen
bereits in den ersten drei Semestern erwerben
und die Priorität zunächst auf die Mathematik legen
oder deren Erwerb auf vier Semester strecken
und dafür dem Zweitfach etwas mehr Priorität geben.
Die erste Möglichkeit hat den Vorteil,
dass dein Studium besser mit dem Verlauf
des Studiums deiner Kollegen im Bachelor of Science harmoniert.
Als Lehramtsstudent besteht dein Studium
auch noch aus dem Zweitfach und Veranstaltungen zur Pädagogik.

Nach bestandenem Bachelorstudium erhältst du den Titel
\glqq Bachelor of Science\grqq\ bzw.\ 
\glqq Bachelor of Arts\grqq\  zusammen mit dem
\glqq Diploma Supplement\grqq, einem langen Begleitheft,
in dem ausführlich drinsteht,
was du während deines Studiums geleistet hast.

Hast du deinen \glqq Bachelor of Science\grqq\ 
in Mathematik erworben, besteht die Möglichkeit,
ein viersemestriges Masterstudium in Mathematik anzuschließen,
das allerdings mit Zulassungsvoraussetzungen verbunden ist.

Hast du als Lehramtsstudent
deinen \glqq Bachelor of Arts\grqq\ erworben,
geht es normalerweise mit dem \glqq Master of Education\grqq\ weiter.
Zusätzlich oder alternativ kannst du auch ein Ergänzungsstudium beginnen,
um zu deinem \glqq Bachelor of Arts\grqq\ 
den \glqq Bachelor of Science\grqq\  in Mathematik zu erwerben.

% Für Lehrämtler geht der erste Studienabschnitt bis zum vierten Semester,
% dann folgt das Praxissemester an einer Schule.
% Danach folgen nochmal fünf Semester an der Uni.
% Im ersten Studienabschnitt ist ziemlich genau festgelegt,
% welche Vorlesungen du besuchen und welche Prüfungen du ablegen musst.
% Hast du alle notwendigen Leistungen erbracht,
% erhältst du deine sogenannte \glqq Zwischenprüfung\grqq.
% Dabei handelt es sich um keine echte Extra-Prüfung,
% sondern nur um eine Ansammlung deiner bisher erbrachten Studienleistungen.

%Da du als Lehrämtler gleich zwei Fächer studieren musst, kann es vorkommen, dass eine der beiden Zwischenprüfungen (eine in Mathematik und eine in dem anderen Fach) erst nach dem fünften Semester abgeschlossen wird. Maximal hast du sechs Semester Zeit, deine Zwischenprüfung abzulegen; ansonsten verlierst du bundesweit den Prüfungsanspruch und wirst exmatrikuliert.

\subsection{Module}
Das Studium ist in sogenannte Module eingeteilt.
Ein Modul kann eine oder mehrere Veranstaltungen wie
z.B.\ eine Vorlesung, ein Seminar oder ein Computerpraktikum beinhalten.
Die Module werden in der Regel benotet.
Zusätzlich bekommt man für jedes bestandene Modul,
unabhängig von der Note,
die ihm zugeordneten \glqq Leistungspunkte\grqq (LP).
Damit wird der geschätzte Arbeitsaufwand
zum Bewältigen des Moduls gewürdigt.
Ein Leistungspunkt entspricht ungefähr 30 Stunden.
Ein Modul mit 9 LP bedeutet also 270 Stunden Arbeit.
Dies schließt z.B.\ den Besuch der Vorlesung, Nacharbeitung,
Prüfungsvorbereitung etc.\ mit ein.

Zum Erwerb des Bachelorgrades benötigt man zusammen mindestens 180 LP.
Für die Lehramtsstudenten fallen während des Bachelorstudiums
78 LP bis 84 LP für den Mathematikteil an,
je nachdem ob du in Mathematik oder in deinem Zweitfach
deine Bachelorarbeit schreiben möchtest.
18 LP sind für die Bildungswissenschaften reserviert,
der Rest ist für das Zweitfach.

Im Masterstudiumsabschnitt fallen nochmals 120 LP an.
Für die Lehramtsstudenten sind davon 77 LP für die Fächer,
sowie 27 LP und 16 LP für Bildungswissenschaft bzw.\ Schulpraxis.

\subsection{Vorlesungen}

Die Vorlesungen sind die zentralen Lehrveranstaltungen an Universitäten.
Dort stellt euch der Dozent mathematische Lehrsätze und
Rechenmethoden vor, führt Beweise aus, und eure Aufgabe ist es,
dazu Aufschriebe anzufertigen und so gut wie möglich
die Gedankengänge zu verfolgen.
Im Grundstudium sitzt ihr dabei zu Hundertschaften in einem Hörsaal.
%% Wenn Ihr später speziellere Vorlesung hört,
%% kommt es vor, daß ihr zu fünft in einer Vorlesung hockt.

In 90 Minuten kann eine ganze Menge Stoff vorgetragen werden
und zwar meist deutlich mehr, als ein durchschnittlicher Student
in dieser Zeit vollständig verdauen kann.
Falls du in einer Vorlesung nicht alles auf Anhieb verstehst,
hin und wiedermal den Faden verlierst,
ist das noch kein Weltuntergang;
vielen anderen Hörern geht es genauso.
Die Vorlesungen müsst ihr daher regelmäßig,
je nach Geschmack für euch allein
oder in kleinen Gruppen, nacharbeiten.

Die Arbeit alleine hat den Vorteil,
daß jeder in seinem Tempo lernen kann
und man weniger abgelenkt wird
($\Rightarrow$ \glqq Lernflow, Effizienz\grqq).
Die Gruppenarbeit hat den Vorteil,
daß man auch mit den Fragen der Kollegen konfrontiert wird,
und Dinge so erläutern muss, dass einen die Kollegen auch verstehen.
So kann man nebenbei seinen Lernerfolg überprüfen
($\Rightarrow$ \glqq Interaktion, Qualitätskontrolle\grqq).
Jeder muss da seine optimale Mischung finden.

Häufig tauchen während einer Vorlesung Fragen auf:
der eine Beweisschritt ist unklar,
die durchgeführte Rechnung ist nicht nachvollziehbar,
oder die Rechnung ist zu Ende, aber du siehst das Ergebnis noch nicht.
In diesen Fällen ist es erlaubt den Dozenten zu fragen.
Meist freuen sich die Dozenten über Nachfragen
und ermutigen daher auch die Hörerschaft dazu.
Solche Chancen lohnt es sich zu nutzen,
denn ein bisschen Interaktion macht die Vorlesung
auch stets etwas lebendiger!
Vereinzelt wird dabei auch ein harmloser Schreibfehler aufgedeckt,
der die Hörerschaft durcheinander gebracht hat.

\vspace{2cm}

\begin{center}
\includegraphics[width=12cm]
{afs/.stud.mathe/fsmath/gemeinsame_Bilder/Comics/labyrinth_puzzle}
\end{center}

\subsection{Übungen und Scheine}

Zu den Vorlesungen werden Übungen angeboten.
Diese dienen zur Vertiefung der Vorlesungsinhalte,
zur Übung der erlernten Rechen-
oder Beweisführungsmethoden, sowie dem Scheinerwerb.
Um an der Prüfung des zugehörigen Moduls teilnehmen zu können,
musst du nämlich in der Regel einen Schein erwerben,
indem du erfolgreich an einer Übungsgruppe teilnimmst.
Zu dieser muss man sich stets anmelden.

Die zum Scheinerwerb notwendigen Leistungen werden stets
{\it Scheinbedingungen} oder {\it Scheinkriterien} genannt.
Um diese zu erfüllen musst du in der Regel einen gewissen Prozentsatz
von Aufgaben erfolgreich bearbeitet haben,
die wöchentlich vom Dozenten oder
einem Assistenten veröffentlicht werden.

Den Nachweiß erbringst du zum Beispiel durch:\\[6pt]
- {\bf Schriftliche Abgaben}\\[2pt]
- {\bf Bearbeitung von Präsenzaufgaben, Kurztests}\\[2pt]
- {\bf Vorrechnen an der Tafel}\\[2pt]
- {\bf Aktive Mitarbeit in der Übungsgruppe}\\
\hspace*{0.5cm}(vor allem regelmäßige Teilnahme)\\[6pt]
Manchmal wird zusätzlich eine Scheinklausur geschrieben,
deren Bestehen notwendig für den Erhalt des Scheins ist.

{\it Die genauen Scheinbedingungen legt der jeweilige Dozent fest.
Welche Bedingungen für euch gelten,
und auch wie man sich für die Übungsgruppen anmeldet,
erfährst du in den ersten Vorlesungen des Semesters.}

Die Übungsgruppen werden je nach Dozent/Assistent unterschiedlich organisiert.
Die beiden häufigsten Arten sind:\\[6pt]
- {\bf Votierübungen}\\[2pt]
- {\bf Präsenzübungen}\\[6pt]
In einer {\it Votierübung} wird zu Beginn
stets ein Votierzettel verteilt,
auf dem die Teilnehmer der Übung ankreuzen (=\glqq votieren\grqq),
welche Aufgaben sie erfolgreich bearbeitet haben.
Der Übungsgruppenleiter, auch Tutor genannt,
wählt dann aus der Liste jemandem zum Vorrechnen aus,
der dann seinen Lösungweg an der Tafel präsentiert.
In der Regel sollte man 2-3 mal vorgerechnet haben
um einen Schein zu erhalten.

In der {\it Präsenzübung} könnt ihr euch wie ihr wollt
in Kleingruppen zusammen setzen und euch daran machen,
das Aufgabenblatt der Woche zu bearbeiten.
Dabei schlendert der Tutor durch den Raum,
schaut euch ein bisschen über die Schulter
und kann euch auch die ein oder andere Hilfestellung geben.

Während der Übungsgruppe habt ihr reichlich Gelegenheit
Fragen zu stellen, die euch der hilfsbereite
und kompetente Tutor gerne beantwortet,
oder die in der Gruppe diskutiert werden können.
Schriftliche Abgaben und Kurztests werden in der Regel
vom Tutor korrigiert und ihr erhaltet eure Abgaben
mit Bewertung eine Woche später wieder zurück.

Eine Übungsgruppe besteht in der Regel aus 15 bis 25 Studenten.
In den Übungsgruppen herrscht also meist eine lockerere Atmosphäre.
Man lernt sich untereinander schneller kennen
als in Vorlesungen und stellt eher mal eine Frage.
Daher können sie für ein erfolgreiches Studium enorm hilfreich sein.

Was kann man mit einem Schein anfangen?
Will man die Prüfung zu einem Modul ablegen,
ist der Schein meist die Zulassungsvoraussetzung zur Prüfung.
Aber selbst wenn nicht, empfehlen wir {\bf dringend},
zu jeder besuchten Vorlesung auch den zugehörigen Schein zu erwerben.
Hast du einmal einen Schein in einer Übung erworben,
aber möchtest die zugehörige Prüfung erst später schreiben,
z.B.\ ein Jahr später, so kannst du ihn dann anerkennen lassen
und die Prüfung mitschreiben.

\subsection{Proseminare und Hauptseminare}
\label{Proseminar}
Jeder muss im Laufe seines Studiums Seminare besuchen.
Die Studenten des Bachelor of Science
machen ein Proseminar und ein Hauptseminar. 
Lehramtsstudenten im Bachelor of Arts
machen ein Proseminar.

Ein Seminar läuft folgendermaßen ab:
Jeder Teilnehmer bekommt einige Seiten
aus einem Buch oder ein Thema mit Literaturliste zugeteilt.
Diese muss er dann gründlich durch- und zu einem Vortrag ausarbeiten.
Jede Woche im Semester steht dann einer dieser Teilnehmer
vorne an der Tafel und hält seinen Vortrag.
Der betreuende Dozent hört sich diese Vorträge
an und stellt zwischendurch ein paar Fragen
zum Verständnis des Stoffes.
Natürlich sollte der Vortragende diese Fragen
dann auch möglichst souverän beantworten können.
Es kann außerdem eine schriftliche Ausarbeitung
des Themas verlangt werden. 
Wenn der Dozent mit dem Vortrag
und der Ausarbeitung zufrieden war und wenn man nicht
allzu oft im Semester gefehlt hat,
besteht man das Modul, bekommt die Leistungspunkte gutgeschrieben
und eine Note (es gibt keine extra Prüfung).

Pro- und Hauptseminare unterscheiden sich nur im Schwierigkeitsgrad. 
In der Regel findet am vorletzten Mittwoch
in der Vorlesungszeit (Aushänge beachten!)  eine Seminarplatzvergabe statt,
auf der die Seminare und fachdidaktischen Übungen des folgenden
Semesters vorgestellt und die Plätze vergeben werden.

\subsection{Computermathematik}

Die Studenten des Bachelor of Science
müssen im ersten Semester die Veranstaltung
\glqq Mathematik am Computer\grqq~besuchen.
Hier erlernt man Basistechniken (Unix, \LaTeX, ...)
und eine Einführung in Mathematiksoftware (Maple, Mathematica, Matlab, ...).
Zudem wird es (als Block am Ende des Semesters)
einen \glqq Programmierkurs\grqq\ geben.
Die Teilnahme an diesem ist eine Zulassungsvoraussetzung
für die Teilnahme an der Modulprüfung Mathematik am Computer Prüfung.

Zusammen mit der Vorlesung \glqq Numerische Lineare Algebra\grqq
~im zweiten Semester bilden diese drei Veranstaltungen das Modul
\glqq Grundlagen der Computermathematik\grqq.

Im fünften Semester gibt für die Studenten
des Bachelor of Science das \glqq Computerpraktikum Mathematik\grqq.
Im Verlauf dieses Praktikums müsst ihr in Dreiergruppen
drei Aufgabenstellungen mit Hilfe
von C++, MatLab oder anderen Programmen bearbeiten.

Für die Lehramtsstudenten im Bachelor of Arts
gibt es das Modul angewandte Mathematik.
Dieses besteht aus den Veranstaltung \glqq Wahrscheinlichkeitstheorie\grqq,
\glqq Numerische Lineare Algebra\grqq\ und dem \glqq Programmierkurs\grqq.

\subsection{Prüfungen}

Die meisten Module schließt ihr am Ende
des Semesters durch die Teilnahme
an der Modulprüfung ab (die ihr hoffentlich besteht).
Die Prüfung findet schriftlich oder mündlich statt
und dauert in der Regel 120 Minuten bzw.\ 30 Minuten.
Sie ist bestanden, wenn man die Note 4,0 oder besser hat.
Falls man durchgefallen ist -- d.h. man hat die Note 5,0 --
muss man die Prüfung wiederholen.

\subsubsection{Zulassungsvoraussetzungen}\label{sssec:zu}

Um zur Modulprüfung zugelassen zu werden,
brauchst du in der Regel den Übungsschein
der entsprechenden Übungsgruppe.
In den meisten Fällen wirst du diesen nicht in Papierform brauchen.
Falls doch kannst du dir diesen bei den Assistenten der Vorlesung abholen.


\subsubsection{Prüfungsanmeldung}\label{sssec:pa}

Zu Prüfungen müsst ihr euch stets rechtzeitig anmelden.
Die Anmeldung zu einer Prüfung erfolgt
für die Studenten des Bachelor of Science
über das LSF
und für die Lehramtsstudenten im Bachelor of Arts
über das C@MPUS-Managementsystem.

Eure Anmeldung müsst ihr innerhalb
des {\it Prüfungsanmeldezeitraums} erledigen.
Dieser ist geht im ersten Semester
vom {\bf 18.~November bis zum 10.~Dezember}.

{\bf LSF:}
\begin{enumerate}
\item
Besuche die Seite {\small\verb|lsf.uni-stuttgart.de|}.
\item
Logge dich mit deiner Kennung matXXXXX und deinem Passwort ein.
\item
Klicke auf \glqq Prüfungsanmeldung und -rücktritt\grqq,\\
in der Menuleiste auf der linken Seite.
\end{enumerate}
In der Liste, die rechts erscheint
kannst du nun das Modul auswählen,
zu dessen Prüfung du dich anmelden willst.\\
{\it Die Anmeldung zur Prüfung über das LSF ist verbindlich.
Innerhalb einer Frist vor der Prüfung
kann man sich ohne Angabe von Gründen abmelden.
Beachte dazu auch die Hinweise im LSF-Portal.}

{\bf C@MPUS:}\\
Man muss sich Einloggen und
so etwas ähnliches wie im LSF machen.
Weitere Details werden noch bekannt gegeben,
da die Autoren bisher niemanden kennen, der sich über C@mpus
zu einer Klausur angemeldet hat.

\subsubsection{Orientierungsprüfung}\label{sssec:or}

Für die Studenten des Bachelor of Science
zählen die Prüfungen zu zwei der Module Analysis I \& II
und LAAG I \& II als Orientierungsprüfung,
für Lehrämtler im Bachelor of Arts die Prüfung im Modul LAAG I.

Orientierungsprüfung heißt: die genannten Module
musst du bis spätestens zum Beginn der Vorlesungszeit
des vierten Semesters erfolgreich bestanden haben.
Keine Panik, das schaffst du schon,
viele vor dir haben ihre Prüfungen
schließlich auch geschafft.

{\it Vergleiche mit der Prüfungsordnung Bachelor of Science Mathematik} §6~(1) in:\\
{\small
\verb|http://www.uni-stuttgart.de/zv/bekanntmachungen/bekanntm_27_2012.pdf|}\\
Beziehungsweise in der {\it Prüfungsordnung Bachelor of Arts Lehramt} §6~(1) in:\\
{\small
\verb|http://www.uni-stuttgart.de/zv/bekanntmachungen/bekanntm_55_2015.pdf|}\\
Sowie 8.~§1 auf S.\ 16 im {\it besonderen Teil} der {\it PO für BA Lehramt} in:\\
{\small
\verb|http://www.uni-stuttgart.de/zv/bekanntmachungen/bekanntm_56_2015.pdf|}

\subsubsection{Wiederholung von Prüfungen}\label{sssec:wi}

{\bf Im Bachelor of Science Mathematik:}\\
Bestandene Prüfungen können grundsätzlich nicht wiederholt werden.
Jede nichtbestandene Prüfung darf einmal wiederholt werden.
Als Termin für deine Wiederholungsprüfung
musst du allerspätestens den übernächsten Prüfungstermin wahrnehmen.
Häufig wird eine Wiederholungsklausur angeboten.

Außerdem darf man im Laufe seines Studiums
in maximal vier Fällen eine nichtbestandene Prüfung
ein zweites Mal wiederholen (gilt nicht für die Module,
die zur Orientierungsprüfung gehören).
Der letzte Rettungsanker:
Fällt man durch die schriftliche Wiederholungsprüfung
eines Orientierungsprüfungsfachs
oder durch die zweite schriftliche Wiederholungsprüfung
einer anderen Vorlesung, so erfolgt zeitnah eine 20-30 Minuten
dauernde mündliche Fortsetzungsprüfung (hier entscheidet sich die Note
nur noch zwischen 4,0 oder 5,0).
Diese Prüfung sollte man bestehen, ansonsten verliert man seinen Prüfungsanspruch
für den Bachelor in Mathematik
in ganz Deutschland und wird exmatrikuliert.

{\it Vergleiche auch mit der Prüfungsordnung BSc Mathematik} §18:\\
{\small
\verb|http://www.uni-stuttgart.de/zv/bekanntmachungen/bekanntm_27_2012.pdf|}

{\bf Im Bachelor of Arts Lehramt Mathematik:}\\
Jede nichtbestandene Prüfung darf einmal wiederholt werden,
danach gibt es eine mündliche Nachprüfung.
Desweiteren kann man drei Prüfungen aus dem Mathematikteil
und zwei aus dem Teil Bildungswissenschaften,
die keine Orientierungsprüfungen sind,
jeweils zweimal wiederholen.
Die Wiederholungsprüfung muss {\it innerhalb
von zwei Semestern} abgelegt werden.
Fällt man durch eine Orientierungsprüfung
oder das zweite mal durch eine andere Prüfung,
muss man {\it zeitnah} zur einer mündlichen Prüfung antreten,
bei der man nur noch Aussicht auf die Noten 4,0 oder 5,0 hat.
Besteht man auch diese nicht,
so verliert man seinen Prüfungsanspruch.

{\it Vergleiche auch mit der Prüfungsordung Lehramt Mathematik} §20:\\
{\small
\verb|http://www.uni-stuttgart.de/zv/bekanntmachungen/bekanntm_55_2015.pdf|}

Wie ihr seht, ist der Prüfungsmodus sehr human.
Es bedarf viel, um endgültig durchzufallen.
Trotzdem solltest du es nicht darauf anlegen.
Zum einen gehen viele eurer Prüfungsergebnisse
auch in eure Endnote mit ein,
zum anderen stehen die Ergebnisse
in eurem \glqq Diploma Supplement\grqq.
Wir raten auch davon ab, Prüfungen aufzuschieben.
Die Arbeitslast wird dadurch immer größer.

\subsection{Fristen und Studienplanung}\label{ssec:fp}

Hier nochmal eine kurze Zusammenfassung
aller prüfungsrechtlich relevanten Fristen
in deinem Studium.\\[6pt]
- {\bf Orientierungsprüfung:}\\
\hspace{1cm}{\it Bis spätestens zum Beginn des 4.~Semesters} (siehe \ref{sssec:or})\\[3pt]
- {\bf Abschluss deines Bachelors:} (d.h.\ {\it das Bestehen aller Module})\\
\hspace{1cm}{\it Bis spätestens zum Ende des 10.~Semesters}\\[3pt]
- {\bf Prüfungswiederholung im Bachelor of Science Mathematik:}\\
\hspace{1cm}{\it Spätestens beim übernächsten möglichen Termin}
(siehe \ref{sssec:wi})\\[3pt]
- {\bf Prüfungswiederholung im Bachelor of Arts Lehramt:}\\
\hspace{1cm}{\it Spätestens innerhalb von zwei Semestern}
(siehe \ref{sssec:wi})\\[3pt]
- {\bf Prüfungsmeldezeitraum:} (siehe \ref{sssec:pa})\\[6pt]
Solange du dich an diese Fristen hältst,
kannst du dein Studium sonst theoretisch völlig frei gestalten.
Musst/Möchtest du zum Beispiel dein Studienleben
durch zusätzliche zeitintensive Hobbies bereichern
(Segelfliegen, Musik, Marathonlaufen, Familie, Geld verdienen, etc.),
so spricht nichts dagegen dies tun.

Der Studienplan liefert dir eine Orientierung,
in welcher Reihenfolge du die Vorlesungen besuchen solltest.
Wirklich relevant für dein Planung ist,
welche Module Voraussetzung für andere Module sind,
zu denen du eine Prüfung schreiben willst
und welche Module obligatorisch,
d.h.\ ein bisschen wichtiger sind.

Solltest du eine oder mehrere Prüfungen nicht bestehen
(selbstverständlich wirst du alles daran legen,
dass es nicht soweit kommt)
musst du deine Studienplanung eventuell anpassen.

Beziehst du das BAföG, so ist zu beachten,
dass dieses in der Regel nur bis zum Ende
des 6.~Semesters gezahlt wird,
d.h.\ solange du in der sogenannten
\glqq Regelstudienzeit\grqq\ bleibst.

Vor Beginn eines jeden Semesters kannst du dir schon überlegen,
welche Vorlesungen du besuchen möchtest.
Dazu gibt es einige Informationsmöglichkeiten:

1. {\bf Die Mathematik-Homepage {\verb|www.mathematik.uni-stuttgart.de|}:}
Über\\ 
$ \sim ${\verb|/fachbereich/studium/vorlesungen/index.html|}\\
kommst du schnell an alle wichtigen Informationen zu den Vorlesungen.
Mit \glqq Studium\grqq\ $\Rightarrow$ \glqq Lehrveranstaltungen\grqq 
kannst du dich auch zu diesen Infos durchklicken.
Falls du weiter in die Zukunft blicken willst, kannst du auch\\
$ \sim ${\verb|/fachbereich/studium/vorlesungen/vorlesungsplanung/index.html|}\\
besuchen.


2. {\bf Das LSF-Portal:}\\
{\verb|lsf.uni-stuttgart.de|}\\
Links in der Menüleiste auf \glqq VVZ + Semester\grqq\ klicken,
der Rest erklärt sich von selbst.
Dort kannst du auch unter \glqq Modulhandbuch
und Prüfungsordnung\grqq\ nachschlagen.
Dort findet man unter Anderem
auch die empfohlenen inhaltlichen Voraussetzungen
sowie die prüfungsrechtlich bindenden
Zulassungsvoraussetzungen zu den Prüfungen der Module.

3. {\bf Das C@mpus-Portal:}\\
{\verb|campus.uni-stuttgart.de|}\\
Um hier die Vorlesungen einzusehen, gehst du auf deine Visitenkarte.
Dein Weg setzt sich mit einem Klick auf Studienstatus fort und mit
eine weiteren Klick auf deinen Studiengang bist du am Ziel.

\newpage
\subsection{Prüfungsausschussvorsitzende und andere wichtige Personen}

Bei allen Fragen zur Prüfungsordnung
kann dir der Vorsitzende des Prüfungsausschusses weiterhelfen.
Das ist für den Bachelor Mathematik:\\
Herr Apl.~Prof.~Dr.~Christian Hesse\\
Zimmer: 8.317\\
Telefon: (0711) 685 - 65343\\
E-Mail: Christian.Hesse@mathematik.uni-stuttgart.de

Die Studiendekanin für den  Bachelor ist
Frau Prof. Dr. Uta Freiberg\\
Zimmer: 8.550\\
Telefon (0711) 685 - 66647\\
E-Mail:  uta.freiberg@mathematik.uni-stuttgart.de

\label{LAzust}
Für das Lehramt gibt es einen eigenen Prüfungsausschussvorsitzenden:\\
Herr Prof. Dr. Jürgen Pöschel
Zimmer:  8.560\\
Telefon: (0711) 685 - 65523\\
E-Mail: Juergen.Poeschel@mathematik.uni-stuttgart.de

Der Studiendekan fürs Lehramt ist
Herr Prof. TeknD Timo Weidl\\
Zimmer: 8.347\\
Telefon (0711) 685 - 65534\\
E-Mail: Timo.Weidl@mathematik.uni-stuttgart.de

Falls du auf Lehramt studierst
und nebenbei mit nur wenig Aufwand
zusätzlich einen Bachelor machen möchtest,
dann wende dich an\\
Herr apl. Prof. Kimmerle\\
Zimmer: 7.344\\
Telefon: (0711) 685 - 65323\\
E-Mail: Wolfgang.Kimmerle@mathematik.uni-stuttgart.de

Bei Fragen oder Unklarheiten zum Bachelor
oder Lehramt kann man sich
bei der Studiengangsmanagerin melden:\\
Frau Dr. Friederike Stoll
Zimmer:  7.553 \\
Telefon: (0711) 685 - 65515 \\
E-Mail: stoll@mathematik.uni-stuttgart.de

{\it Auch zu finden auf der Mathematik-Homepage unter \glqq Fachbereich\grqq\ 
$\Rightarrow$ \glqq Ansprechpartner\grqq\ etc...}
\\
\subsection{Tipps und Tricks für das Mathematikstudium}
Das Wichtigste wollen wir dir nicht vorenthalten,
das Mathestudium wird kein Spaziergang.
Aber wir wollen dich auch nicht beunruhigen,
sondern dich vor Fallen warnen, in die viele von uns getappt sind.
Wir raten dir:
\begin{itemize}
\item
Nicht abschrecken lassen, wenn es nicht auf Anhieb klappt.
Achte mal darauf, dass es den meisten so geht.

\item
Aus diesem Grund gemeinsam arbeiten.
Im Gegensatz zu dem Klischee ist es wichtig untereinander zu kommunizieren.

\item
Frag so oft und so viel wie du kannst.
Habe keine Angst, dass deine Frage zu einfach ist.
Aus der höheren Perspektive ist sie das meistens.

\item
Du solltest Zeit für Übungsaufgaben einplanen.
Wichtig ist, dass die meiste Zeit für Aufgaben und nicht für die
Vorlesung drauf geht.
\end{itemize}
Leider gibt es kein allgemeines Kochrezept um am Ball zu bleiben,
dies musst du für dich selbst rausfinden.
Den Meisten hilft es jedoch erfahrungsgemäß
\begin{itemize}
\item
die Vorlesung zu besuchen und nachzuarbeiten.
\item
Übungsblätter nicht einfach abzuschreiben.
\end{itemize}
Das mit dem Abschreiben ist fundamental wichtig,
sonst wirst du höchstwahrscheinlich bei den ersten Scheinklausuren
ziemlich auf die Schnauze fliegen.
Aber vergiss auch nicht,
dass es seine Zeit braucht,
um sich an die Denkweise in der Mathematik zu gewöhnen.

Auch wichtig ist,
sich von dem Gesagten nicht abschrecken zu lassen und gelassen zu bleiben.
Wahrscheinlich begibst du dich in eine dir unbekannte Welt,
dass dies meistens nicht reibungslos abläuft sollte klar sein.

Wir wissen, dass wir uns wiederholen.
Aber es ist wichtig sich nicht unterzukriegen zu lassen,
dann wirst du eine Menge Spaß mit der Mathematik haben.
  
\newpage
\vspace*{4cm}
\ifthenelse{\boolean{online}}
{

}
{
\begin{center}
  \includegraphics[width=12cm]{afs/.stud.mathe/fsmath/gemeinsame_Bilder/Comics/comic41}
\end{center}
}
\newpage

%%%%%%%%%%%%%%%%%%%%%%%%%%%%%%%%%%%%%%%%%%
%% Stand: 03. Oktober 2016
%%%%%%%%%%%%%%%%%%%%%%%%%%%%%%%%%%%%%%%%%%
