\subsubsection{Nebenfach Wirtschaftswissenschaften}

Die Vorlesungen zu den Wirtschaftswissenschaften
finden an der Uni in der Stadtmitte statt,
man muss also zwischen den Vorlesungen hin- und herpendeln.
So kommt es vor, dass sich manche Mathematik-
und BWL-Vorlesungen überschneiden.
Das ist allerdings nicht so schlimm,
wie man als Studienanfänger vielleicht meinen könnte,
kann aber durchaus lästig sein.
In diesem Fall ist es am besten,
Arbeitsgruppen zu bilden und die Mathematik-
und BWL-Vorlesungen abwechselnd zu besuchen.
Meist geben die Dozenten ein Skript heraus
oder halten sich an ein Buch.
Empfehlenswert ist es, am Anfang des Semesters
im Internet nachzuschauen,
wann die jeweiligen Vorlesungen beginnen.
Die BWLer haben nämlich die Angewohnheit,
meist erst in der zweiten oder gar dritten Semesterwoche
mit den Vorlesungen anzufangen -- dafür hören sie dann evtl.\ eine Woche vor
Semesterende auch schon wieder auf!\\[-6ex]

\begin{center}
\begin{tabular}{|@{}c@{}|@{}c@{}|@{}c@{}|@{}c@{}|@{}c@{}|} 
\multicolumn{1}{c}{\makebox[3cm]{1}}
&\multicolumn{1}{c}{\makebox[3cm]{}}
%&\multicolumn{1}{c}{\makebox[3cm]{}}
&\multicolumn{1}{c}{\makebox[3cm]{4}}
&\multicolumn{1}{c}{\makebox[3cm]{5}}
\\[0.2cm] 

\cline{1-1}\cline{3-4}
\bf Grundl.~BWL&& \bf BWL~II& \bf BWL~III\\
2+1~|~\it3 LP&&  2$\times$(2+1)-\it9 LP&2$\times$(2+1)-\it9 LP\\
\cline{1-1}\cline{3-4}
\bf Grundl.~WiWi&\multicolumn{3}{c}{}\\
2+1~|~\it3 LP&\multicolumn{3}{c}{}\\
\cline{1-1}
\end{tabular}
\end{center}

Alle Module schließen mit einer Prüfung ab. Es gilt:\\[0.5ex]
\begin{tabular}{lcrl}
Nebenfachnote & = &$\frac{3}{24}\,\cdot$&Grundlagen der BWL\\[0.5ex]
              & + &$\frac{3}{24}\,\cdot$&Grundlagen der Wirtschaftswissenschaften\\[0.5ex]
              & + &$\frac{9}{24}\,\cdot$&BWL~II\\             
              & + &$\frac{9}{24}\,\cdot$&BWL~III
                                  
\end{tabular}

%%%%%%%%%%%%%%%%%%%%%%%%%%%%%%%%%%%%%%%%%%%%%%%%%%%%%%%%%%%%%
%% Stand: 31. August 2010
%%%%%%%%%%%%%%%%%%%%%%%%%%%%%%%%%%%%%%%%%%%%%%%%%%%%%%%%%%%%%

