\subsection{Schlüsselqualifikationen}

Von Bachelor-Studenten wird erwartet,
dass sie sogenannte `Schlüsselqualifikationen'
(besser bekannt unter dem Namen `Softskills')
im Umfang von 18 LP erwerben.

Ein paar dieser Punkte erwirbst du automatisch
durch das Computerpraktikum.
Dies ist nämlich genau genommen kein Mathematikmodul,
sondern seine 6 LP gehören den Softskills an.

Von den restlichen 12 LP müssen genau 6 LP
in einer fachaffinen Veranstaltung erworben werden.
Dies sind zum Beispiel:\\[6pt]
- {\bf `Präsentation und Vermittlung von Mathematik'} (3 LP)\\[2pt]
- {\bf `Computertutorium Mathematik'} (3 LP)\\[2pt]
- {\bf Nebenfach-Module} (3 - 6  LP)\\[6pt]
Genau 6 LP müssen aus sogenannten fachübergreifenden
Schlüsselqualifikationen stammen.
Ein entsprechender Katalog für diese uniweit angebotenen
Veranstaltungen wird rechtzeitig bekannt gegeben.
Die Anmeldung erfolgt auch hier über das LSF
in einem gewissen Zeitraum.

\vspace*{2cm}
\begin{center}
\includegraphics[width=\textwidth]
{/afs/.stud.mathe/fsmath/gemeinsame_Bilder/Comics/comic435}
\end{center}

%%%%%%%%%%%%%%%%%%%%%%%%%%%%%%%%%%%%%%%%%%%%%%%%%%%%
%% Stand: 31. August 2010
%%%%%%%%%%%%%%%%%%%%%%%%%%%%%%%%%%%%%%%%%%%%%%%%%%%%
