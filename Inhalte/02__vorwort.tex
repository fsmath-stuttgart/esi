\vspace*{15cm}
{\small
\section*{Impressum} 
\begin{tabular}{@{}ll@{}} 
Herausgeber: & Fachgruppe Mathematik der Universität Stuttgart \\
 & Pfaffenwaldring 57\\
 & 70569 Stuttgart\\
 & Tel. (07 11) 685-653 41\\
 & E-Mail: fachgruppe@mathematik.uni-stuttgart.de\\
 & im Internet {\tt http://www.stud.mathematik.uni-stuttgart.de/Fachgruppe/}\\
V. i. S. d. P.:    & Gentian Rrafshi\\
Redaktion und Layout: & Gentian Rrafshi, Andre Zieher, Tillmann Kleiner\\
Titelbild: & Jörg Hörner, Jim Magiera \\
Druck: & Fachbereich Mathematik \\
Auflage: & 102 \\
Erscheinungsdatum: & 7. Oktober 2015 \\ \\
\multicolumn{2}{@{}l@{}}{\normalsize \it Alle Angaben in diesem Heft sind ohne
 Gewähr!}
\end{tabular}}

\newpage
\section*{Vorwort}
Hallo!

Willkommen an der Universität Stuttgart
und willkommen im Kreis der Mathe\-matikstudenten.
Bald geht dein Mathematikstudium los,
und wahrscheinlich hast du wenig Ahnung, 
wie dieses so ablaufen soll, welche Vorlesungen du
hören musst und und und\dots

Damit du dein Studium nicht ganz so hilf\-los beginnen musst,
geben wir für dich dieses Informationsheft heraus.
Hier steht das Wichtigste drin,
was du über dein Bachelorstudium als Mathematikstudent
oder Lehramtsstudent mit dem Fach Mathematik wissen solltest.
(Die hier enthaltenen Informationen haben wir
den jeweiligen Studienplänen und
Prüfungsordnungen nach bestem Wissen und Gewissen entnommen.
Die offiziellen Informationsquellen haben wir stets mit angegeben.
Das entbindet dich aber natürlich nicht,
dich auch selber zu informieren.
Insbesondere sind alle Angaben hier ohne Gewähr.)

%% Beachte bitte auch, dass wir am
%% {\bf \ESETagEins. und \ESETagZwei.~Oktober} eine
%% {\bf Erstsemestereinführung} veranstalten,
%% während der wir dir die wichtigsten Informationen
%% über dein Studium geben werden und du
%% uns noch alle offenen Fragen stellen kannst.

Wenn du Fragen hast oder in lockerer Atmosphäre
ein bisschen über das Studium plaudern willst,
schau doch einfach bei uns im Fachgruppenzimmer vorbei
- du bist immer herzlich willkommen!

\begin{flushright}{ \it Deine Fachgruppe Mathematik}
\end{flushright}
\vspace*{\fill}
{\small P.S.:\\
1.) Alle Personenbezeichnungen werden in diesem Heft
in männlicher Form verwendet,
beziehen sich aber selbstverständlich
auf alle Personen unabhängig vom Geschlecht.\\[2pt]
2.) Die Lehramtsstudenten nach GymPO
(\glqq altes\glqq\ modularisiertes Lehramt)
bekommen von uns die Version des ESI des WS2014/15.
Denn im GymPO hat sich von 2014 auf heute nichts geändert.\\[2pt]
3.) Am Dienstag Abend findet übrigens beinahe Regelmäßig
unser Spieleabend statt.
Auch der ist ein Besuch wert.}

%%%%%%%%%%%%%%%%%%%%%%%%%%%%%%%%%%%%%%%%
%%% Stand: 23. Juni 2010
%%%%%%%%%%%%%%%%%%%%%%%%%%%%%%%%%%%%%%%%
