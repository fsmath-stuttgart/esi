\section{Studienplan für das Lehramt}
\label{LA}
Den Studienplan für dein Zweitfach
hier aufzuführen, würde den Rahmen
dieses Infoheftes sprengen,
darum beschränken wir uns auf die
von dir verlangten Leistungsnachweise in Mathematik.
Falls du wichtige Fragen zum Lehramtsstudium haben solltest,
gehe bitte zum Lehramtszuständigen (siehe \ref{LAzust})
oder schau bei uns in der Fachgruppe vorbei.\\

Die Mathematikmodule sind fast immer die selben Module
wie im Bachelor oder Master.
Eine echte Ausnahme bilden die Veranstaltungen zur Fachdidaktik,
die speziell für die Lehramtsstudenten angeboten werden.
In der Fachdidaktik sollst du lernen,
mathematische Inhalte schülergerecht zu vermitteln.

% Aufschlüsselung LP

\subsection{Grundstudium bis zur Zwischenprüfung}

Lehrämtler müssen in ihrem Grundstudium folgende Vorlesungen besuchen:
\begin{itemize}
\item Analysis I (9 LP)
\item Analysis II (9 LP)
\item Lineare Algebra und Analytische Geometrie I (LAAG I) (9 LP)
\item Lineare Algebra und Analytische Geometrie II (LAAG II) (9 LP)
\item Numerik für Lehramtsstudenten (4 LP)
\item Fachdidaktik I (6 LP)
\end{itemize}

Zu jedem dieser Module (außer eventuell der Fachdidaktik)
musst du eine schriftliche Prüfung ablegen.
Prüfungsvorleistung ist immer der jeweilige Schein.
Das Modul \glqq Numerik für Lehramtsstudenten\grqq~
besteht aus der Vorlesung \glqq Numerische Lineare Algebra\grqq~
und einem Programmierkurs.

\subsubsection{Orientierungsprüfung}

Deine Orientierungsprüfung besteht aus der Modulprüfung
zu einer der folgenden Vorlesungen:
Analysis I, Analysis II, LAAG I oder LAAG II.

\subsubsection{Zwischenprüfung}

Die \glqq Zwischenprüfung\grqq~in der Mathematik
ist keine eigentliche Prüfung, 
sondern nur eine Ansammlung von erbrachten Studienleistungen.
Für deine Zwischenprüfung musst du 30 LP
aus den oben genannten Grundstudiumsvorlesungen gesammelt haben.
Diese muss bis zum Beginn der Vorlesungszeit
des siebten Semesters abgelegt sein.\\
\textbf{Achtung:}
Du musst trotzdem alle oben genannten
Grundstudiumsmodule erfolgreich absolvieren!



\subsection{Hauptstudium bis zum 1. Staatsexamen}

Im Hauptstudium musst du folgende Module erfolgreich abschließen:
\begin{itemize}
 \item Algebra und Zahlentheorie (9 LP)
 \item Analysis III (9 LP)
 \item Fachdidaktik II (4 LP)
 \item Geometrie (6 LP)
 \item Wahrscheinlichkeit und Statistik (9 LP)
\end{itemize}

Zusätzlich dazu musst du noch 18 Leistungspunkte
aus den Aufbau-, Ergänzungs-, Seminar- und Vertiefungsmodulen
aus den Bachelor- und Masterstudiengängen sammeln
- natürlich nicht die, die oben schon genannt sind!
Wie die Prüfungen genau aussehen
und ob ein Schein Prüfungsvorleistung ist,
erfährst du vom jeweiligen Dozenten zu Beginn der Veranstaltung. \\
Außerdem brauchst du ein Seminar im Umfang von 3 LP.\\ 
Als Lehramtsstudent musst du in einem deiner Fächer
(in welchem kannst du selbst entscheiden)
eine wissenschaftliche Arbeit schreiben.
Dafür erhältst du 20 LP. 

\subsection{1. Staatsexamen}

Am Ende deines Studiums legst du in jedem Fach
eine mündliche Prüfung über 60 Minuten ab.
Dafür bekommst du nochmals je 10 LP.
Für die Prüfung in Mathematik wählst du in Absprache
mit deinen Prüfern drei Schwerpunktbereiche
aus den folgenden Themengebieten (bezieht sich nicht auf konkrete Module!) aus:

\begin{itemize}
\item Algebra oder Zahlentheorie
\item Analysis
\item Geometrie
\item Numerische Mathematik
\item Stochastik
\end{itemize}


\subsection{Spezielle Veranstaltungen fürs Lehramt}

\subsubsection{Bildungswissenschaften und EPG}

Als Lehrämtler musst du auch
einige Pädagogikveranstaltungen besuchen.
Dazu gehören:
\begin{itemize}
\item Das Bildungswissenschaftliche Begleitstudium (insgesamt 18 LP), bestehend aus:
\end{itemize}
\begin{center}
\begin{tabular}{|p{17em}|c|c|c|c|c|c|c|c|} \hline
Modul \hfill Semester & 1 & 2 & 3 & 4 & 5 & 6 & 7 & LP \\  \hline
Lehren \& Lernen & x & x & &&&&& 6 \\  \hline
Entwicklung, Lernen und Vermittlung &&&x & x &&&& 6 \\ \hline 
Erziehung \& Bildung &&&&&& x & x & 6 \\ \hline
\end{tabular}
\end{center}

\begin{itemize}
\item Das Ethisch-philosophische Grundlagenstudium, kurz: EPG (insgesamt 12 LP), bestehend aus:
\begin{itemize}
\item EPG-1-Seminar (Grundkurs Ethik) mit 6 LP. Empfohlen wird EPG-1 als Block direkt nach dem Praxissemester (von Weihnachten bis Ende Wintersemester).
\item EPG-2-Seminar (Fach- \& Berufsethik) mit 6 LP. Empfohlen wird EPG-2 im 8. Semester, also nach dem Praxissemester (EPG-1 als Voraussetzung vorteilhaft, jedoch nicht notwendig.)
\end{itemize}


\item Das Modul Selbst- \& Sozialkompetenz aus dem Bereich Personale Kompetenz (insgesamt 6 LP), das in zwei Seminare geteilt wird. Den ersten Teil sollte man vor dem Praxissemester (5. Semester), den zweiten Teil während oder nach dem Praxissemester hören.
\end{itemize}

\subsubsection{Orientierungspraktikum}

Das zweiwöchige Orientierungspraktikum muss vor dem Studium,
spätestens jedoch bis zum Beginn
des dritten Semester absolviert werden.
Du besuchst in dieser Zeit eine Schule (nicht deine eigene)
und kannst so einen Eindruck gewinnen,
ob dir der Lehrerberuf gefallen könnte.
Für das Orientierungspraktikum erhältst du keine Leistungspunkte,
da es sich streng genommen um eine Zulassungsvoraussetzung
fürs Studium handelt. \\
Zum Orientierungspraktikum muss man sich anmelden,
nähere Informationen findest du unter:\texttt{http://www.orientierungspraktikum-bw.de/}

\subsubsection{Schulpraktikum}
Das Schulpraktikum ist ein eigenes Modul mit 16 LP.
Es findet in einem ungeraden Semester
(vorgesehen ist das fünfte Semester,
also nach der Zwischenprüfung) statt
und geht über 13 Wochen ab Schulbeginn im September.
\linebreak In dieser Zeit besuchst Du eine Schule,
beobachtest den Unterricht von anderen Lehrern
(man sagt auch \glqq hospitieren\grqq)
und darfst auch selbst erste Unterrichtserfahrung sammeln. \\
Das Schulpraxissemester kann bestanden
oder auch nicht bestanden werden.
Wenn du das Praktikum nicht bestehst,
kannst du es einmal wiederholen.
Auch zum Schulpraktikum musst du dich anmelden:\\

%\vspace*{0.5 cm}
Der Anmeldezeitraum ist vom ersten Schultag
nach den Osterferien bis zum 15.05.,
mind. jedoch 4 Wochen, und erfolgt in
zwei Schritten:
\begin{center}
\begin{itemize}
  \item
    Seminarplatz für das Praxissemester an einer Schule im Internet reservieren. 
    Eine Reservierung an mehreren Schulen ist nicht gleichzeitig möglich.
  \item
    Irgendwann meldet sich dann die Schule. Meistens muss man noch zu einem
    persönlichen Gespräch dorthin, das aber nur dazu dient, sich schon mal
    kennenzulernen.
\end{itemize}
\end{center}
Unter \verb|http://www.praxissemester-bw.de/| oder
natürlich bei uns in der Fachgruppe findest du weitere Informationen.
Es gibt auch die Möglichkeit, das Orientierungs-
und Schulpraktikum im Ausland zu absolvieren.
Details findest du auch dazu im Internet.
Wichtig ist in diesem Fall,
sich frühzeitig zu informieren.

%\begin{center}
%  \includegraphics[width=\textwidth]{/afs/.stud.mathe/fsmath/gemeinsame_Bilder/Comics/certainty.eps}
%\end{center}


%%%%%%%%%%%%%%%%%%%%%%%%%%%%%%%%%%%%%%%%%%
%% Stand: 31. August 2010
%%%%%%%%%%%%%%%%%%%%%%%%%%%%%%%%%%%%%%%%%%
