\subsubsection{Nebenfach Technische Mechanik} 

Technische Mechanik (TM) als Nebenfach bildet
eine Brücke zu den Ingenieurwissenschaften.

Mechanik unterteilt sich in die Kinematik,
die Lehre von den Bewegungen, 
und die Dynamik, die Lehre von Kräften.
Die "`technische"' Mechanik beschränkt sich
auf die Anwendungen in der Technik,
d.h.\ Biomechanik oder Himmelsmechanik
werden zum Beispiel nicht behandelt.

Natürlich ist die Mechanik ein Teilgebiet der Physik.
Dies bedeutet jedoch nicht,
dass man Vorkenntnisse aus der Physik mitbringen 
oder gar ein gro"ses Interesse an der Physik haben muss.

Die Vorlesungen geben einen schönen Einblick
in die Arbeitsweise der Ingenieure.
Hier werden "`real existierende"' Probleme analysiert
und in ein mathe\-mati\-sches Modell übertragen,
d.h.\ man versucht das Problem mit mathematischen Gleichungen
darzustellen und mit deren Hilfe zu lösen.
Damit ist auch schon gesagt,
dass die TM sehr viel mit Mathematik zu tun hat. 
Wer also nicht nur Mathematik abstrakt lernen,
sondern auch mal anwenden will,
für den ist TM als Nebenfach sehr zu empfehlen. 

Nebenbei sei noch erwähnt, dass die Mechanik-Institute
der Universität an Mathematikern interessiert sind.
Mathematik-Studierende sind dort zwar seltene,
aber gern gesehene Gäste.

%\begin{center}
%\begin{tabular}{|@{}c@{}|@{}c@{}|@{}c@{}@{}c@{}|@{}c@{}@{}c@{}} 
%\multicolumn{1}{c}{\makebox[2.4cm]{1}}&\multicolumn{1}{c}{\makebox[2.4cm]{2}}&
%\multicolumn{1}{c}{\makebox[2.4cm]{3}}&\multicolumn{1}{c}{\makebox[2.4cm]{4}}&
%\multicolumn{1}{c}{\makebox[2.4cm]{}}&\multicolumn{1}{c}{\makebox[2.4cm]{}}\\[0.2cm] 
%\cline{1-4}
%\bf TM~I&\multicolumn{2}{c|}{\bf TM~II+III}&\bf TM~IV~M&&\\
%\it6 LP&\multicolumn{2}{|c|}{\it12 LP}&\it6 LP&\\
%\cline{1-4}
% \end{tabular}
%\end{center}

\begin{center}
	\begin{tikzpicture}
		\colnr{0}{1}{1}
		\colnr{1}{1}{2}
		\colnr{2}{1}{3}
		\colnr{3}{1}{4}
	    \modul{0}{0}{TM~I\\6 LP}
	    \doublemodul{1}{0}{TM~I+II\\12 LP}
	    \modul{3}{0}{TM~III\\6 LP}
	    
		
    \end{tikzpicture}
\end{center}
Aus den Modulen \glqq Technische Mechanik I\grqq
~und \glqq Technische Mechanik II+III\grqq~benötigt man einen Schein,
das Modul \glqq Technische Mechanik IV
für Mathematiker\grqq~hat am Ende eine Prüfung.
Die Note dieser Prüfung ist die Nebenfachnote.
Bei allen Fragen zu TM ist Prof.~Peter Eberhard
(\texttt{eberhard@itm.uni-stuttgart.de}) gerne bereit, dir weiterzuhelfen.


%\vspace{3cm}
%\begin{center}
%\epsfig{file=hagar.ps,width=14cm}
%\end{center}
