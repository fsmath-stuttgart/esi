\subsection{Gleichstellung}

Wie auch in einigen anderen Fachbereichen
ist der Frauenanteil in der Fakultät Mathematik
leider immer noch gering:
bei den Studierenden sind zwar ca. $50\,\%$  weiblich,
doch dieser Anteil wird immer niedriger,
je höher es in der Hierarchie geht.
Mit dieser Problematik beschäftigt sich -- fächerübergreifend --
die Gleichstellungsbeauftragte der Universität Stuttgart:
sie setzt sich dafür ein, dass gezielt Maßnahmen unternommen werden,
um die Studien-- und Arbeitssituation für Frauen an der
Universität zu verbessern und so den Frauenanteil allmählich zu erhöhen.
Dafür hat sie in jeder Fakultät eine/n Ansprechpartner/in
(meist eine Assistentin/wissenschaftliche Mitarbeiterin),
die speziell die Studentinnen und Mitarbeiterinnen
des jeweiligen Fachbereiches informiert und berät.
In der Fakultät Mathematik und Physik ist dies zur Zeit\\[2ex]
\hspace*{4cm} Dr.-Ing. Helga Kumric\\
\hspace*{4cm} Zimmer 3.159 \\
\hspace*{4cm} Tel. 0711/ 685-62197 \\
\hspace*{4cm} E-Mail h.kumric@physik.uni-stuttgart.de\\[2ex]
Bei Fragen, Problemen, Anregungen etc.\ 
zu diesem Bereich ist das also eine Anlaufstelle. 

\vspace{1cm}
\begin{center}
\includegraphics[width=\textwidth]
{/afs/.stud.mathe/fsmath/gemeinsame_Bilder/Comics/dating_pools}
\end{center}

%%%%%%%%%%%%%%%%%%%%%%%%%%%%%%%%%%%%%%%%%%%%%
%%% Stand: 01. Juli 2010
%%%%%%%%%%%%%%%%%%%%%%%%%%%%%%%%%%%%%%%%%%%%%
