\section{Studienplan für den Bachelor of Science}

Die Module des Bachelorstudiengangs
teilen sich in drei Gruppen auf.
Es gibt Mathematikmodule, Nebenfachmodule (24 LP)
und Schlüsselqualifikationen (sog. \glqq Soft-skills\grqq, 18 LP).

\subsection{Mathematikmodule}
Zunächst einmal die Auflistung aller Mathematikmodule
im Laufe deines sechs-semestrigen Bachelorstudiums
(das Computerpraktikum zählt offiziell als Schlüsselqualifikation).
Die Semesterangaben im folgenden sind stets
nicht prüfungsrechtlich bindend.

%\vspace*{-0.5cm}

\begin{center}
\includegraphics[width=17cm]
{afs/.stud.mathe/fsmath/gemeinsame_Bilder/Friederike/Bachelor.pdf}
\end{center}

Der obige Plan gibt dir eine Orientierung darüber,
in welcher Reihenfolge du die Vorlesungen besuchen solltest.
Solltest du z.B.\ Analysis~I nicht beim ersten Versuch bestehen,
heißt das nicht, daß du erst ein Jahr aussetzen musst,
bevor mit Analysis~II anfangen kannst.
Du versucht einfach die wichtigsten Inhalte von Analysis~I
soweit nachzuholen, daß du in Analysis~II noch mitkommst.
Möglicherweiße bestehst du dann Analysis~II
und schreibst dann deine Wiederholungsprüfung
für Analysis~I einfach im 3.~Semester.
Diese Möglichkeit soll verhindern,
daß sich dein Studium nicht unnötig verzögert.

Beachte stets die Zulassungsvoraussetzungen
für spätere Module (siehe \ref{sssec:zu}).

\subsubsection{Der Grundlagenteil (typischerweise 1. - 3. Semester)}

Der Grundlagenteil besteht aus folgenden {\bf Pflichtmodulen}:\\[6pt]
- {\bf Analysis~I-III}\\[2pt]
- {\bf Lineare Algebra und Analytische Geometrie~I-II}  (LAAG I-II)\\[2pt]
- {\bf Grundlagen der Computermathematik}\\[6pt]
Das Modul \glqq Grundlagen der Computermathematik\grqq
~geht über zwei Semester und setzt sich aus der 1+1 Veranstaltung
\glqq Mathematik am Computer\grqq ~im ersten Semester,
einem \glqq Programmierkurs\grqq, der als Blockveranstaltung
in der vorlesungsfreien Zeit stattfindet,
und der 2+1 Lehrveranstaltung \glqq Numerische Lineare Algebra\grqq
~im zweiten Semester zusammen.

Außerdem besucht man im dritten Semester noch zwei {\bf Basismodule}.
Dazu wählt man zwei aus folgenden drei Modulen:\\[6pt]
- {\bf Topologie}\\[2pt]
- {\bf Numerische Mathematik~I}\\[2pt]
- {\bf Wahrscheinlichkeitstheorie}\\[6pt]
Die Module werden in der Regel
durch eine schriftliche Prüfung (120 min) abgeschlossen.
Für die drei Module werden Analysis~I+II
und LAAG~I+II als Voraussetzung empfohlen.
%braucht man für die ersten beiden Analysis- bzw.\ LAAG-Vorlesungen
%jeweils den dazugehörigen Übungsschein.
Für das Modul \glqq Grundlagen der Computermathematik\grqq
\ braucht ihr die Scheine des Programmierkurs
%und der Veranstaltung \glqq Mathematik am Computer\grqq
und die Prüfung in \glqq Numerische Lineare Algebra\grqq.

Zwei Prüfungen der Module \glqq Analysis I \& II\grq
~und \glqq LAAG I \& II\grqq~dienen zugleich
als sogenannte {\it Orientierungsprüfung}.
Sie müssen bis zum Beginn der Vorlesungszeit des vierten Fachsemester
erfolgreich bestanden sein.

{\bf Achtung:}
Viele Vorlesungen aus dem zweiten Studienabschnitt erfordern
bestimmte Basis- oder Aufbaumodule als Voraussetzung.
Ihr legt also bei der Wahl der Basismodule
schon eine thematische Ausrichtung eures weiteren Studiums fest.
Auf der anderen Seite kann euch niemand davon abhalten
zusätzliche Vorlesungen zur Ergänzung zu hören,
falls ihr zum Beispiel eure thematische Ausrichtung wechseln wollt.

\newpage
\subsubsection{Der Vertiefungsteil (typischerweise 4. - 6. Semester)}

Ihr müsst unter anderem
drei der folgenden fünf {\bf Aufbaumodule} absolvieren:\\[6pt]
- {\bf Algebra}\\[2pt]
- {\bf Geometrie}\\[2pt]
- {\bf Höhere Analysis}\\[2pt]
- {\bf Numerische Mathematik~II} (Voraussetzung: Numerische Mathematik~I)\\[2pt]
- {\bf Mathematische Statistik} (Voraussetzung: Wahrscheinlichkeitstheorie)\\[6pt]
Am Ende dieser Aufbaumodule findet jeweils eine Prüfung statt.
Wie diese genau abläuft, wird vom Dozenten zu Beginn festgelegt. 

Der Vertiefungsteil dient zudem der Spezialisierung.
Dazu besucht ihr\\
  - {\bf Vertiefungsmodule}\\
  - ein {\bf Computerpraktikum Mathematik} und\\
  - ein {\bf Mathematisches Seminar}.

Das \glqq Mathematische Seminar\grqq\ 
besteht aus einem Proseminar und einem Hauptseminar.
Das Proseminar wird meist im dritten oder vierten Semester besucht,
das Hauptseminar im fünften oder sechsten Semester.
Im Gegensatz zu den meisten Vorlesungen
werden Pro- und Hauptseminare jeweils
in den Sommer- {\it und} Wintersemestern angeboten.

Habt ihr euch durch das Bestehen von Modulen
genügend Leistungspunkte (90 LP) erarbeitet,
dürft ihr mit eurer Bachelorarbeit beginnen.
Hier sollt ihr weitgehend selbstständig
eine umfangreiche, aber klar abgesteckte
mathematische Aufgabenstellung schriftlich ausarbeiten.

\subsubsection{Endnote}

Eure Endnote in der Mathematik
setzt sich aus dem nach Leistungspunkten
und Studienabschnitt gewichteten (Analysis 1-3, LAAG 1-2,
Grundl. der Computermathematik, Computerpraktikum und Basismodule ~=~Faktor~1; \\
Aufbau-, Vertiefungsmodule und Seminar ~=~Faktor~2; Bachelorarbeit~=~Faktor~3)
Durchschnitt der Noten der einzelnen Mathematikmodule zusammen
und wird auf eine Stelle nach dem Komma gerundet.

%\begin{tabular}{lcrl}
%  Mathenote & = & $\frac{9}{231} \, \cdot$ & (Analysis I + Analysis II + Analysis III)\\[0.5ex]
%            & + & $\frac{9}{231} \, \cdot$ & (LAAG I + LAAG II + Basis I + Basis II)\\[0.5ex]
%            & + & $\frac{6}{231} \, \cdot$ & (Grundlagen der Computermathematik)\\[0.5ex]
%            & + & $\frac{18}{231}\, \cdot$ & (Aufbau I + Aufbau II + Aufbau III + Aufbau IV)\\[0.5ex]
%            & + & $\frac{18}{231}\, \cdot$ & Vertiefung\\[0.5ex]
%            & + & $\frac{12}{231}\, \cdot$ & (Ergänzung + Computerpraktikum + Seminar)\\[0.5ex]
%            & + & $\frac{36}{231}\, \cdot$ & Bachelorarbeit\\
%\end{tabular}

Eure {\bf Gesamtnote} setzt sich zu $13/15$ aus der Mathenote
und zu $2/15$ aus der Nebenfachnote zusammen.
Bei Wahl von speziellen Vertiefungsmodulen ist die Gesamtzahl
der Leistungspunkte nicht 180, sondern 183 oder 186.

\begin{center}
\includegraphics[width=11cm]
{afs/.stud.mathe/fsmath/gemeinsame_Bilder/Comics/comic2}
\end{center}

\subsection{Nebenfachmodule}

Zusätzlich zu den Mathematikmodulen
musst du eines der nachfolgenden Nebenfächer besuchen.
Die hier angegebenen Semester der jeweiligen Veranstaltungen
sind nur Empfehlungen, keine Pflicht. 
Soweit es möglich war, haben wir angegeben,
welchen Umfang (Vorlesung + Übung) die jeweiligen Veranstaltungen haben.
Wenn diese Angabe fehlt, stand diese zum Druckzeitpunkt noch nicht fest. 

Die Nebenfächer sind eine schöne Abwechslung
zu den Mathematikvorlesungen und können sehr viel Spa"s machen.
Such dir also dein Nebenfach gut aus,
damit es nicht zur Qual wird.

{\bf Achtung:}
Bei manchen Nebenfächern entscheidet eine einzige Prüfung
über eure Nebenfachnote.
Diese Prüfung zählt in diesem Fall
dann mehr zu eurer Endnote als die Bachelorarbeit!

%%%%%%%%%%%%%%%%%%%%%%%%%%%%%%%%%%%%%%%%%%%%%%%%%%%%%%%
%%  Stand: 31. August 2010%%%%%%%%%%%%%%%%%%%%%%%%%%%%%%%%%%%%%%%%%%%%%%%%%%%%%%%
